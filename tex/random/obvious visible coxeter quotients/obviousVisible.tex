\documentclass{article}

\usepackage{amsthm}

\begin{document}
\section{Visible Quotients}
Let $L$ be a link and let $D$ be a diagram of $L$.
A \textbf{Coxeter quotient} is a quotient $\pi_1/H$ of $\pi_1(S^3 \setminus L)$ that is isomorphic to a Coxeter group $W$ with generating set $S$, where the isomorphism between $\pi_1/H$ maps meridians of $L$ to reflections (conjugates of elements in $S$) in $W$. A Coxeter quotient is called \textbf{visible} in $D$ if for all $s \in S$ there exists a pre-image $s' \in \pi_1$ that is a Wirtinger generator, and it is called \textbf{obvious} if one can see by hand that these Wirtinger generators $s'$ indeed generate $\pi_1$.

The following is a procedure to algorithmically find all obvious visible Coxeter quotients in a given diagram $D$. First, find minimal sets of obvious generators. This can be done in the following way. Take an arbitrary set $S$ of Wirtinger generators. Mark the corresponding arcs in the diagram. Then repeat the following. At each crossing where the over-arc and one of the under-arcs are marked, mark the second under-arc. Then $S$ generates $\pi_1$ if and only if at some point all arcs in the diagram are marked.

Having determined all minimal sets $S$ of obvious generators, do the following for each of them. Use the Word Problem for Coxeter groups to try to deduce from a given Coxeter matrix the defining relations of $\pi_1$ with respect to this set of generators (it suffices to consider prime entries to exclude many of them). 

This brute forces your way into obtaining all obvious visible Coxeter quotients with respect to a given diagram.
\end{document}