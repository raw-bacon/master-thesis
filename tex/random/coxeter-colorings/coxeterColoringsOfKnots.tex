\documentclass{article}

\usepackage{amsmath}
\usepackage{amsthm}
\usepackage{amsfonts}
\usepackage{graphicx}

\newtheorem{theorem}{Theorem}[section]
\newtheorem{lemma}[theorem]{Lemma}
\newtheorem{proposition}[theorem]{Proposition}
\newtheorem{conjecture}[theorem]{Conjecture}

\theoremstyle{definition}
\newtheorem{definition}[theorem]{Definition}
\newtheorem{example}[theorem]{Example}
\newtheorem{remark}[theorem]{Remark}

\title{Coxeter Colorings of Knots}
\author{Levi}
\date{May 2019}

\begin{document}
\maketitle
\tableofcontents

\section{Knot Theory}

A \textbf{knot} is an embedding $K: S^1 \rightarrow S^3$. Abusing notation, we frequently write~$K$ for the image of $K$. We will frequently consider the complement of the set $K$ in $S^3$, so we will write $K^c = S^3 \setminus K$.

A \textbf{diagram} of $K$ is a planar graph $D \subset S^2$ that is the projection of $K$ onto a 2-plane, in which the over- and undercrossings are marked such that $K$ can be reconstructed from $D$. The diagram $D$ is called \textbf{regular} if all the intersections of $D$ are transversal and have order two. 

\subsection{Invariants}
Diagrams allow us to define \textbf{knot invariants}, which are mathematical objects assigned to either knots or diagrams. In the case that they are assigned directly to the knots, they are expected to be invariant under the following equivalence relation: Two knots $K$ and $K'$ are equivalent if and only if there is an isotopy of $S^3$ between the identity and a homeomorphism $S^3 \rightarrow S^3$ mapping $K$ to $K'$. If the invariants are assigned to a knot diagram, they are expected to be invariant under Reidemeister moves. 

An easy example of a knot invariant is the \textbf{crossing number}, which is defined to be the minimal number $c(K)$ of crossings over all diagrams of $K$. Knot tables are generally sorted according to this invariant, because it introduces a crude but reliable measure of complexity of a knot. In a similar spirit we can define the \textbf{bridge index} of $K$ to be half the least possible number of critical points of a smooth function $S^3 \rightarrow \mathbb{R}$ restricted to $K$. The bridge number of $K$ will be denoted $b(K)$. We will also consider the \textbf{unknotting number} $u(K)$ which is the least number of crossings over all diagrams $D$ that need to be inverted in order to transform $D$ into a diagram of the \textbf{unknot}, corresponding to a diagram with no crossings.

\subsection{The Plan}
A curve $\gamma: S^1 \rightarrow K^c$ is called a \textbf{meridian} if it is a generator of the first homology $H_1(K^c)$ over the integers $\mathbb{Z}$. For this to make sense, recall that Alexander duality implies that $H_1(K^c)$ is infinite cyclic. It can be shown that the \textbf{knot group} of $K$, defined to be the fundamental group $\pi_K = \pi_1(K^c)$ of its complement, is generated by meridians. This can be seen using the Wirtinger presentation, which gives an explicit generating set consisting of meridians. The least number of meridians in a presentation of $\pi_K$ is called the \textbf{meridional rank} of $K$ and will be denoted $mr(K)$. We denote the set of equivalence classes of meridians in the knot group $\pi_K$ by $M_K$.

In this text we are interested in homomorphisms $\pi_K \rightarrow G$ for groups $G$ we understand well enough to arrive at some conclusions about $K$. The main motivation to do this is the following conjecture of Cappell and Shaneson.

\begin{conjecture}[Meridional Rank Conjecture]
For any knot $K$, the bridge number and the meridional rank of $K$ are equal. In symbols, $b(K) = mr(K)$.
\end{conjecture}

Another motivation to consider these homomorphisms will be covered soon. It is an interesting relationship between so-called colorings and the unknotting number.

\section{Dihedral Colorings}
Before considering general Coxeter colorings we first recap the Fox $n$-colorings. Later on, this will be a special case of Coxeter colorings. 

The $\textbf{dihedral group}$ of order $2n$ is the group with presentation $$D_n = \langle a, b \; | \; a^2 = b^2 = 1, (ab)^n = 1 \rangle.$$
It can be shown that $D_n$ is the group of symmetries of a regular $n$-gon. The set of \textbf{reflections} of $D_n$ is the set $R_n$ of conjugates of $a$ or, equivalently, of $b$. Note that $R_n$ is the set of elements $r^ka$, where $0 \leq k < n$ and $r = ab$ is a rotation generating the set of orientation preserving symmetries of the regular $n$-gon on which $D_n$ acts.

\subsection{Algebraic Miracles}\label{sec: miracles}
A homomorphism $\pi_K \rightarrow D_n$ mapping the set of meridians $M_K$ to the set of reflections $R_n$ is traditionally called a \textbf{Fox $n$-coloring}. We will also refer to such homomorphisms as \textbf{dihedral colorings} in this text. Denote the set of all valid $n$-colorings of a knot $K$ by $\text{col}_n(K)$.

\begin{lemma}\label{lem:conjugation-invariance-dihedral}
Let $K$ be a knot. Then the set of colorings $\textup{col}_n(K)$ is closed under simultaneous conjugation by elements of $D_n$.
\end{lemma}

\begin{proof}
Let $e_1, \dots, e_k$ denote the arcs of a diagram $D$ of the knot $K$. Consider a coloring $\pi_K \rightarrow D_n$ given by $e_i \mapsto g_i.$ Then the $g_i$ satisfy the Wirtinger relations of $D$, whence so do the $hg_ih^{-1}$ for any $h \in D_n$.
\end{proof}

\begin{proposition}\label{prop:col-vector-space}
The set $\textup{col}_n(K)$ of $n$-colorings of a knot $K$ forms a module over the ring $\mathbb{Z}/n\mathbb{Z}$. In particular, if $n$ is a prime $p$ then $\textup{col}_p(K)$ is a vector space over the field $\mathbb{F}_p$.
\end{proposition}

\begin{proof}
Label the arcs of any diagram $D$ of $K$ with symbols $e_1, \dots, e_k$. Consider the identification $R_n \rightarrow \mathbb{Z}/n\mathbb{Z}$ given by $r^ia \mapsto i$. Then the set $col_n(K)$ of colorings of $K$ can be viewed as a subset of the module $(\mathbb{Z}/n\mathbb{Z})^k$ interpreted as the space of all possible $n$-colorings of $D$. We will now show that it is indeed a submodule.

Consider a crossing in $D$ as with overstrand $s$ and understrands $t, u$. Then, in $D_n$ we have the relation $u = sts$. Under our identification this yields the relation $2s = t + u$ in $\mathbb{Z}/n\mathbb{Z}$. Indeed, using $r^la = ar^{-l}$ we see
\begin{align*}
r^ua & = r^sar^tar^sa  \\
	 & = ar^{-s}r^tr^{-s} \\
	 & = r^{2s -t}a.
\end{align*}
Since this is a linear equation it is preserved under sums, which shows that $\text{col}_n$ identified as a subset of $(\mathbb{Z}/n\mathbb{Z})^k$ is closed under addition. In particular, it is also closed under scalar multiplication, whence it is itself a module.
\end{proof}

In the case that $p$ is prime, the $p$-\textbf{coloring dimension} of $K$ is defined to be the dimension of the vector space $\text{col}_p(K)$ over $\mathbb{F}_p$ and is denoted $\text{cdim}_p(K)$. There is a nice application of the coloring dimension to the unknotting number as follows.

\begin{theorem}
Let $K$ be a knot and let $p$ be prime. Then $\textup{cdim}_p(K) \leq u(K) + 1$.
\end{theorem}
The proof of the theorem relies on the fact that changing a crossing changes the coloring dimension by at most one. 
%TODO really do the proof

\subsection{The Dihedral Coloring Group}\label{sec: dihedral-reformulating}
In order to generalize the above observations to general Coxeter groups, we need a slightly different perspective on what the coloring dimension is. This is due to the fact that the space of Coxeter colorings is in general not a module over any ring. The only thing we will be able to work with is the action of $D_n$ on the set $\text{col}_n(K)$ of colorings by simultaneous conjugation, see Lemma \ref{lem:conjugation-invariance-dihedral}.

Consider a coloring $e_i \mapsto r^{m_i}a$ of a diagram $D$. Let us consider the $D_n$-orbit of this coloring. We saw above that conjugating by an element $r^ua$ has the effect of turning $r^{m}a$ into $r^{2u - m}a$. Similarly, conjugating by $r^u$ has the effect of turning $r^ma$ into $r^{m+2u}a$. So conjugation acts simply, bot not doubly transitive on $\text{col}_n(K)$. 

Write $D_n^+$ for the subgroup of $D_n$ consisting of orientation preserving symmetries. A typical element of $D_n^+$ is of the form $r^l$ for some $l$, unique up to congruence mod $n$.

\begin{proposition}
Let $K$ be a a knot. Any orbit of $\textup{col}_n(K)$ under the action of~$D_n^+$ is of size $n$.
\end{proposition}

\begin{proof}
The stabilizer in $D_n^+$ of a reflection in $D_n$ is just the identity.
\end{proof}


Note that in Proposition \ref{prop:col-vector-space} we established that there is some kind of action of the set of colorings $\text{col}_n(K)$ on itself. More concretely, we saw that if we have colorings $\rho: e_i \mapsto r^{m_i}a$ and $\rho': e_i \mapsto r^{m_i'}a$, then so is $\rho\rho': e_i \mapsto r^{m_i + m_i'}a$. Denote this operation by $(\:\cdot\:): \text{col}(D_n,K) \times \text{col}(D_n,K) \rightarrow \text{col}(D_n,K).$ 

\begin{proposition}
The operation $\cdot$ turns $\textup{col}(D_n, K)$ into a group.
\end{proposition}

\begin{proof}
The set $\text{col}(D_n, K)$ is closed under $\cdot$, and $\cdot$ is evidently associative. The identity in this group is the coloring mapping all meridians to $a$, and the inverse of a coloring $\rho$ is the coloring $e_i \mapsto a \rho(e_i) a$.
\end{proof}

\begin{proposition}\label{prop: dihedral-composition-of-colorings-passes-to-orientation-preserving-orbits}
Let $K$ be a knot and let $\rho, \rho': \pi_K \rightarrow D_n$ be colorings. Let $g,h \in D_n^+$ be orientation-preserving. Then, $\rho \cdot \rho'$ and $g\rho g^{-1} \cdot h\rho' h^{-1}$ lie in the same orbit.
\end{proposition}

\begin{proof} Write $\rho: e_i \mapsto r^{k_i} a$ and $\rho': e_i \mapsto r^{k_i'} a$. Also write $g = r^l$ and $h = r^m$. Compute $g \rho(e_i) g^{-1}  = r^{k_i + 2l}a$ and $h \rho'(e_i) h^{-1} =  r^{k_i' + 2m}a$ to see that
\begin{align*}
g\rho g^{-1}\cdot h\rho' h^{-1} (e_i)
& = r^{k_i + k_i' + 2l + 2m}a \\
& = r^{l+m}\rho \cdot \rho' (e_i) r^{-l-m}.\qedhere
\end{align*}
\end{proof}

\begin{remark}
In case $g$ is a rotation and $h$ is a reflection, write $g = r^l$ as above and $h = r^ma$. Compute $g \rho(e_i) g^{-1}  = r^{k_i + 2l}a$ and $h\rho'(e_i)h^{-1} = r^{2m - k_i'}a$. Then, similarly as above, we get $g\rho g^{-1}\cdot h\rho' h^{-1} (e_i)
 = r^{k_i - k_i' + 2l + 2m}a,$
which is conjugate to $r^{k_i - k_i'}$, but not conjugate to $r^{k_i + k_i'}$ for all $i$ in general. So there is no hope of extending Proposition \ref{prop: dihedral-composition-of-colorings-passes-to-orientation-preserving-orbits} to arbitrary $g,h \in D_n$.
\end{remark}

By Proposition \ref{prop: dihedral-composition-of-colorings-passes-to-orientation-preserving-orbits}, the operation $\text{col}(D_n, K) \times \text{col}(D_n, K) \rightarrow \text{col}(D_n, K)$ passes to the orbit space $\text{col}_n(K)/D_n^+$.
Equipping this with the operation $\cdot$ we obtain a group
$\text{cgp}(D_n, K) = \text{col}_n(K)/D_n^+,$ called the $n$-\textbf{coloring group} of $K$.

\begin{conjecture}
The coloring dimension of $K$ is equal to the number of elements of the coloring group $\textup{cgp}(D_n, K)$ needed to generate $\textup{cgp}(D_n,K)$.
\end{conjecture}


If this is true then we can redefine the $n$-\textbf{coloring dimension} of $K$ to be the least possible number of generators of the coloring group $\text{cgp}(D_n,K)$.

\subsection{The Dihedral Coloring Matroid}
We want to define a closure operator $\langle \: \cdot \: \rangle : 2^{\text{col}(D_n, K)} \rightarrow 2^{\text{col}(D_n, K)}$ such that whenever $K$ is a knot for which $S \subset \text{col}(D_n, K)$ consists of valid colorings, then so does $\langle S \rangle$. Additionally, $\langle \: \cdot \: \rangle$ satisfies the matroid axioms. Moreover we want that $\langle \: \cdot \: \rangle$ satisfies that $\langle S \rangle$ is the set of all colorings such that whenever $K$ is $S$-colorable, then it is $\langle S \rangle$-colorable.

\section{Symmetric Colorings}
In this section we try to illustrate the purpose of the reformulation done in Section \ref{sec: dihedral-reformulating}. We consider a particularly famous class of groups within the family of Coxeter groups, namely the symmetric groups.

\subsection{Starting Over}
The \textbf{symmetric group} of a set $X$ is the group of its bijections. In the case that $X$ is a finite set consisting of $n$ elements, we write $X = \{1, \dots, n\}$ and denote the symmetric group of $X$ by $S_n$. An element of $S_n$ is disjoint by disjoint cycle notation. E.g., the elements $(i \; j)$ for $1 \leq i < j \leq n$ are called \textbf{transpositions}. Note that $S_n$ is generated by $n-1$ particular transpositions, namely those of the form $a_i = (i \;\; i + 1)$ for $1 \leq i < n$. Note that $a_i$ and $a_j$ satisfy the braid relation $a_ia_ja_i=a_ja_ia_j$ if $|i - j| = 1$, and they commute if $|i - j| > 1$. Moreover, all the $a_i$ are of order two.

A homomorphism $\pi_K \rightarrow S_n$ mapping the set of meridians $M_K$ to the set of transpositions $R_n$ is called an $n$-\textbf{symmetric coloring}. Because $S_3$ and $D_3$ are isomorphic via an isomorphism mapping transpositions in $S_3$ to reflections in~$D_3$ we have that a $3$-dihedral coloring is the same as a $3$-symmetric coloring.

\subsection{The Orbits}
Consider the action of $S_n$ on $R_n$ by conjugation. Evidently, the stabilizer of a transposition $(i \;\; j)$ is isomorphic to the group generated by $S_{n-2}$ and $(i \;\; j)$, where $S_{n-2}$ acts on the set $\{1, \dots, n\} \setminus \{i,j\}$. We later on want to inductively describe the orbits, so in order to get the induction started we first consider the case $n = 3$.

An orbit consisting only of constant colorings in $S_3$ always has three elements, namely the constant maps mapping $\pi_K$ to either $(1 \; 2)$, $(1 \; 3)$, or $(2 \; 3)$. If two colors appear in an orbit, then the coloring is automatically surjective and thus attains all three colors. The stabilizer of a transposition, say $(1\;2)$, is the set
$\text{stab}_{(1 \; 2)} = \{(), (1\;2)\}$,
which shows that the orbit has size six.


%TODO
Let us now describe the coloring monoid $\text{cmon}(S_3,K)$. But wait - we have absolutely no idea how.
\section{Coxeter Colorings}
\subsection{The Monoid}
How should the operation $\text{col}(W,K) \times \text{col}(W,K) \rightarrow \text{col}(W,K)$ be defined?

\begin{example}
Let $W$ be a Coxeter group and let $s$ be an arbitrary reflection. That is, $s$ is conjugate to an element in $S$. Consider the operation $$(\rho \cdot_s \rho')(x) = \rho(x) s \rho(x).$$ Then, the set of $W$-orbits of $W^n$, where $n$ is the rank of $W$, forms a group with respect to $(\;\cdot_s)$. This is kind of freaky, but indeed, the identity is the coloring $m \mapsto s$, and the inverse of $\rho$ under $(\cdot_s)$ is $s\rho s$.

But well, let $K$ be a knot and let $\rho, \rho': \pi_K \rightarrow W$ be colorings. Then a Wirtinger relation $x = zyz$ yields the relation $(\rho \cdot_s \rho')(x) = (\rho \cdot_s \rho')(zyz)$, i.e., we want to show that
$$\rho(x)s\rho'(x)\rho(zyz)s\rho'(zyz) = 1$$
in $W$. This doesn't really seem plausible because $\rho(x)s\rho'(x)$ is not necessarily a reflection.
\end{example}

\begin{example}[Other Ideas] This is just brainstorming.
\begin{itemize}
\item Maybe only define $\cdot$ partially.
\item Make the $s$ in the definition of $\cdot_s$ above depend on the colorings $\rho$ and $\rho'$. Does adding the roots work better?
\item Maybe it is wrong to consider the orbits as in Section \ref{sec: dihedral-reformulating}. Maybe it is enough to define the operation.
\item It might be better to try to define a matroid structure on $\text{col}(W,K)$. 
\end{itemize}
\end{example}
\end{document}