\documentclass{article}

\usepackage{amsfonts}
\usepackage{amsthm}
\usepackage{tikz-cd}

\newtheorem{theorem}{Theorem}[section]
\newtheorem{proposition}[theorem]{Proposition}
\newtheorem{lemma}[theorem]{Lemma}
\newtheorem{conjecture}[theorem]{Conjecture}

\theoremstyle{definition}
\newtheorem{recall}[theorem]{Recall}
\newtheorem{remark}[theorem]{Remark}
\newtheorem{example}[theorem]{Example}
\newtheorem{definition}[theorem]{Definition}

\newcommand{\Aut}{\textup{Aut}}
\newcommand{\Homeo}{\textup{Homeo}}

\begin{document}
\section{Covering Spaces}
\begin{recall}
Let $X, Y$ be topological spaces and let $p: Y \rightarrow X$ be a normal covering map. Then, the group $\Aut(p)$ of deck transformations of $p$ is isomorphic to the quotient $\pi_1(X) / p_*(\pi_1(Y))$.

Conversely, suppose we are given a normal subgroup $H$ of the fundamental group $\pi_1(X)$. 
We want to construct a normal covering space $Y$ with a covering map $p: Y \rightarrow X$ such that $\Aut(p)$ is isomorphic to the quotient $\pi_1(X) / H$. 
To do this, let $\widetilde{X}$ be the universal cover of $X$. 
Consider the universal covering map $\widetilde{p}: \widetilde{X} \rightarrow X$ of $X$. 
Then the group $\Aut(\widetilde{p})$ of deck transformations of $\widetilde{p}$ is isomorphic to $\pi_1(X)$. 
Thus, we can consider $H$ as a subgroup of $\Aut(\widetilde{p})$, which is itself a subgroup of the group $\Homeo(\widetilde{X})$.
Consider now the orbit space $Y = H \backslash \widetilde{X}$. 
Because the universal covering map $\widetilde{p}$ is invariant under $H$ we obtain a covering map $p: Y \rightarrow X$. Since $\widetilde{X}$ is simply connected, $\pi_1(Y)$ is isomorphic to $H$. This shows that $\Aut(p)$ is isomorphic to $\pi_1(X)/H$.

The following theorem summarizes what we have shown.
\end{recall}

\begin{theorem}
Let $X$ be a topological space. Then there is a bijection between the set of normal coverings $p: Y \rightarrow X$ of $X$ and normal subgroups $H$ of the fundamental group $\pi_1(X)$ of $X$.
\end{theorem}

\begin{remark}
The construction above would also work if instead of choosing the universal cover $\widetilde{X}$ we chose a space $Z$ for which we have that $\pi_1(Z)$ is a normal subgroup of $H$ under the correct identifications.
\end{remark}

\section{Coxeter Covers}



\end{document}