\documentclass[11pt]{article}
%more features in math mode
\usepackage{amsmath}


%for "here"
\usepackage{float}

%mathcal and stuff
\usepackage{amsfonts}

%seems to not work if removed
\usepackage{amsrefs}

%theorems
\usepackage{amsthm}

%commutative diagrams
\usepackage{tikz-cd}

%draw polygons
\usepackage{tikz}

%include pictures
\usepackage{graphicx}

%subfigures
\usepackage{caption}
\usepackage{subcaption}

%french spacing
\frenchspacing

\hyphenation{para-bolic}



\title{The $\pi$-Orbifold Group}
\date{16. March 2019}


\newtheorem{theorem}{Theorem}[section]
\newtheorem{proposition}[theorem]{Proposition}
\newtheorem{lemma}[theorem]{Lemma}
\newtheorem{corollary}[theorem]{Corollary}
\theoremstyle{definition}
\newtheorem{definition}[theorem]{Definition}
\newtheorem{example}[theorem]{Example}
\newtheorem{remark}[theorem]{Remark}
\newtheorem{conjecture}[theorem]{Conjecture}

\begin{document}
\section{The $\pi$-Orbifold Group}
\begin{definition}
Let $L \subset S^3$ be a link, and let $m \in \pi_1(S^3 \setminus L)$ be a meridian. Then the \textit{$\pi$-orbifold group} of $L$ is
$$O(L) = \pi_1(S^3 \setminus L)/\langle\langle m^2 \rangle \rangle,$$
where $\langle \langle m^2 \rangle \rangle$ is the normal subgroup of $\pi_1(S^3 \setminus L)$ generated by $m^2$. The link $L$ is said to be \textit{sufficiently complicated} if $O(L)$ is infinite.
\end{definition}

\begin{example}
Note that if $K$ is a 2-bridge knot, then $O(K)$ is a dihedral group of finite order and hence finite. There are also 3-bridge knots $K$ for which $O(K)$ is finite: see Table \ref{tab:finite_3-bridge_orbifold_groups}.

\begin{table}[ht]
\centering
\begin{tabular}{c|c}
{$K$} & {$\#O(K)$} \\
\hline
$8_{5}$ & 336 \\
$8_{10}$ & 528 \\
$8_{19}$ & 48 \\
$8_{20}$ & 144 \\$8_{21}$ & 240

\end{tabular}
\caption{Some finite $\pi$-orbifold groups \\ 
of 3-bridge knots with 8 crossings.}
\label{tab:finite_3-bridge_orbifold_groups}
\end{table}
It is not clear whether Table \ref{tab:finite_3-bridge_orbifold_groups} contains all 3-bridge knots with 8 crossings with finite $\pi$-orbifold group.
\end{example}

\begin{definition}
Let $L\subset S^3$ be a link. The \textit{meridional rank} of $L$ is the minimal number of meridians required to generate $\pi_1(S^3 \setminus L)$. The \textit{involutuary rank} of $L$ is the minimal number of involutions required to generate $O(L)$.
\end{definition}

\begin{lemma}\label{lemma:inv_less_mer}
The involutary rank of a link $L \subset S^3$ is bounded from above by the meridional rank of $L$.
\end{lemma}

\begin{proof}
Let $\langle m_1, \dots, m_r | R\rangle$ be a Wirtinger presentation for $L$ such that the meridional rank of $L$ is $r$. Then the equivalence classes of $m_1, \dots, m_r$ in the $\pi$-orbifold group $O(L)$ are involutions that generate $O(L)$.
\end{proof}

\begin{conjecture}
The meridional rank and the involutary rank are equal.
\end{conjecture}

\begin{definition}
A link $L \subset S^3$ is \textit{Coxeter} if $O(L)$ admits a Coxeter quotient of rank $s$, where $s$ is the involutary rank of $L$. Here, a \textit{Coxeter quotient} of $O(L)$ is a quotient of $O(L)$ that is a Coxeter group whose generating set consists of equivalence classes of meridians of $\pi_1(S^3 \setminus L)$. The \textit{Coxeter rank} of $L$ is the maximal rank of a Coxeter quotient of $O(L)$.
\end{definition}

\begin{lemma}
The Coxeter rank of a link $L \subset S^3$ is bounded from above by the involutary rank of $L$.
\end{lemma}

\begin{proof}
Let $q$ be the Coxeter rank of $L$, and let $W = \langle m_1, \dots, m_q | R \rangle$ be a Coxeter presentation of a maximal Coxeter quotient of $O(L)$. Then $W$ is not generated by less than $q$ involutions (consider the fixed subspaces of the generators in the reflection representation of $W$), so neither is $O(L)$. Thus, the Coxeter rank is bounded from above by the involutary rank. \ref{lemma:inv_less_mer}.
\end{proof}

\begin{example}
It is an interesting question whether the Coxeter rank of a link $L\subset S^3$ is always equal to its bridge index. Sadly, this is not always the case: the knot $L = 8_{18}$ has bridge index 3 but Coxeter rank 2. This is because $O(8_{18})$ does not admit irreducible representations into $SO(3)$, see Raphael Zentner: Question 10.2 in \textit{A CLASS OF KNOTS WITH SIMPLE SU(2)-REPRESENTATIONS}.
\end{example}

\end{document}
