\documentclass[11pt]{article}

\usepackage{amsmath}
\usepackage{amsfonts}
\usepackage{amsthm}

\newtheorem{theorem}{Theorem}[section]
\newtheorem{proposition}[theorem]{Proposition}
\newtheorem{conjecture}[theorem]{Conjecture}
\newtheorem{corollary}[theorem]{Corollary}
\newtheorem{question}[theorem]{Question}
\newtheorem{lemma}[theorem]{Lemma}

\begin{document}
\section{Reflection Representation}
Consider a real vector space $V$ of dimension $n$. A \textbf{linear reflection} is a linear map $A \in \text{GL}(V)$ fixing an $(n-1)$-dimensional hyperplane called its \textbf{mirror} and one eigenvector of eigenvalue $-1$, called its \textbf{root}.

Let $(W,S)$ be a Coxeter system of type $M = (m_{ij})$. Then a \textbf{reflection representation} is a linear representation $\rho: W \rightarrow \text{GL}(V)$ obtained as follows.
Writing $S = \{s_1,\dots,s_n\}$, consider a real vector space $V$ of dimension $n$ and a basis $\{e_1,\dots,e_n\}$. Let $B$ be a symmetric bilinear form satisfying
$$B(e_i, e_j) = -\cos(\pi/m_{ij})$$
if $m_{ij} < \infty$. Otherwise, if $m_{ij} = \infty$ we just assume $B(e_i, e_j) \leq 1.$ The flexibility there is useful to guarantee irreducibility (See Davis for why this is allowed). Then define the representation $\rho: W \rightarrow \text{GL}(V)$ by $s_i \mapsto \rho_i$, where
$$\rho_i(x) = x - 2B(x, e_i)e_i.$$
\begin{proposition}
Any reflection representation obtained by the preceeding construction defines a faithful linear representation.
\end{proposition}


\begin{proposition}
If $B$ is non-degenerate then no non-zero point of $V$ is fixed by all reflections.
\end{proposition}

\section{Pretzel Links}
Let $L = P(l_1,\dots, l_n)$ be a Pretzel link. We assume $|l_i| \geq 2$. We associate to $L$ the Coxeter group
$$W = \langle s_1, \dots, s_n \; | \; (s_2s_1)^{l_1}, (s_3s_2)^{l_2}, \dots, (s_ns_{n-1})^{l_{n-1}}, (s_1s_n)^{l_n} \rangle.$$
We will call groups of this form \textbf{Pretzel Coxeter groups}. Their Coxeter matrix is
$$M = \left( \begin{matrix}
1 & l_1 & \infty & \infty & \infty &\cdots & \infty & l_n \\
l_1 & 1 & l_2 & \infty & \infty & \cdots & \infty & \infty \\
\infty & l_2 & 1 & l_3 & \infty &\cdots & \infty & \infty \\
\infty & \infty  & l_3 & 1 & \ddots & \ddots & \vdots & \vdots \\
\infty & \infty & \infty & \ddots & \ddots & \ddots & \infty & \infty \\
\vdots & \vdots & \vdots & \ddots &\ddots & 1 & \l_{n-2} & \infty \\
\infty & \infty & \infty & \cdots & \infty & l_{n-2} & 1 & l_{n-1} \\
l_n & \infty & \infty & \cdots & \infty & \infty  & l_{n-1} & 1
\end{matrix} \right).$$

\begin{proposition}
The Pretzel Coxeter group associated to a Pretzel link $L$ is a quotient of the link group $\pi_1(S^3 \setminus L)$. 
\end{proposition}


\begin{lemma}\label{lem:arthurs-trick}
Let $X,Y$ be real square matrices of the same size, and suppose $X$ is invertible. Then, for all but finitely many $\lambda \in \mathbb{R}$ the matrix $\lambda X + Y$ is invertible.
\end{lemma}

\begin{proof}
The function $\mu \mapsto \det(X + \mu Y)$ is a non-zero polynomial, so $X + \mu Y$ is invertible for almost all $\mu$ in $\mathbb{R}$. Dividing by $\mu$ and letting $\lambda = \pm \mu^{-1}$ we see that whenever $X + \mu Y$ is invertible, so is $\lambda X + Y$.
\end{proof}

\begin{proposition}
Let $L = P(l_1,\dots, l_n)$ be a Pretzel link. Then the Pretzel Coxeter group associated to $L$ admits a linear reflection representation of degree $n$ such that the only point fixed by all reflections is zero.
\end{proposition}

\begin{proof}
This includes multiple applications of Lemma \ref{lem:arthurs-trick}. If $n = 2$, then $B$ is positive definite and thus non-degenerate. If $n = 3$ then $B$ might note be positive definite, but one can easily check with Gauss elimination that $B$ is non-degenerate.

Next, we consider the case $n=4$. It is not strictly necesarry to consider this case separately but it serves to illustrate the strategy to come. Let
$$
X_4 = \left( \begin{matrix}
0 & 0 & 1 & 0 \\
0 & 0 & 0 & 1 \\
1 & 0 & 0 & 0 \\
0 & 1 & 0 & 0 \\
\end{matrix} \right), \;\;\;
Y_4 = \left( \begin{matrix}
1		&		b_1		&		0		&	b_4 	\\
b_1		& 		1		&		b_2			&	0	\\
0 	& 		b_2		&		1			&	b_3		\\
b_4		&		0	&		b_3			&	1		\\
\end{matrix} \right),
$$
where $b_i = -cos(\pi/l_i)$. By Lemma \ref{lem:arthurs-trick} there exists $\lambda \leq -1$ such that the bilinear form $B$ with matrix $\lambda X + Y$ is invertible, in which case the radical of $B$ is trivial. Thus, the reflection representation determined by $B$ is such that no point except zero is fixed. This concludes the proof for $n = 4$.

For $n = 5$ note similarly that the matrix
$$X_5 = \left( \begin{matrix}
0 & 0 & 1 & 1 & 0 \\
0 & 0 & 0 & 1 & 1 \\
1 & 0 & 0 & 0 & 1 \\
1 & 1 & 0 & 0 & 0 \\
0 & 1 & 1 & 0 & 0
\end{matrix} \right)$$
is invertible. Defining $Y_5$ similarly as in the case $n = 4$ yields the desired result.

Now let $n$ be arbitrary.
Define $Z_n$ to be the matrix of the standard bilinear form (introduced e.g. in Humphreys) corresponding to the Coxeter matrix $M$. More concretely we have
$$ Z_n = \left( \begin{matrix}
1	& b_1	& -1	& -1	& -1	& \cdots	& -1	& b_n	\\
b_1 & 1	& b_2	& -1	& -1	& \cdots	& -1	& -1	\\
-1	& b_2	& 1		& b_3	& -1	& \cdots	& -1	& -1	\\
-1	& -1	& b_3 	& 1		& \ddots& \ddots	& \vdots& \vdots\\
-1	& -1	& -1	& \ddots& \ddots& \ddots	& -1	& -1	\\
\vdots&\vdots&\vdots& \ddots& \ddots& 1			& b_{n-2}&x		\\
-1	& -1	& -1	& \cdots& -1	& b_{n-2}	&  1	& b_{n-1}\\
b_n	& -1	& -1	& \cdots& -1	& -1		& b_{n-1}&1		\\

\end{matrix} \right)$$
where $b_i = -\cos (\pi/l_i)$. Let $X$ be the block matrix
$$X_{2k} = \left( \begin{matrix}
0 & I_{n/2} \\
I_{n/2} & 0
\end{matrix} \right), \;\;\;
X_{2k+5} = \left( \begin{matrix}
0 & 0 & I_{k/2} \\
0 & X_5 & 0 \\
I_{k/2} & 0 & 0
\end{matrix} \right)
$$ where $I_m$ is the identity matrix of size $m$ and $X_5$ is as in the case $n = 5$. Then let $Y = Z + X$ and as above apply Lemma \ref{lem:arthurs-trick}.
\end{proof}


\begin{corollary}
The meridional rank of a Pretzel link $L = P(l_1,\dots, l_n)$, where we assume $l_i \geq 2$, is equal to $n$. 
\end{corollary}

\begin{proof}
Since $\pi_1(S^3 \setminus L)$ is generated by the obvious meridians we have that the meridional rank is less than or equal to $n$. Moreover, the meridional rank is bounded from below by $n$ because by irreducibility of the reflection representation the Pretzel Coxeter quotient associated to $L$ is not generated by less than $n$ reflections. Thus the meridional rank is equal to $n$.
\end{proof}

\begin{conjecture}
Any other Coxeter quotient of $\pi_1(S^3 \setminus L)$ is a quotient of the Pretzel Coxeter group.
\end{conjecture}

Sadly this is not true, because the $(3,3,3)$-Pretzel knot has an $S_3$-quotient.

\begin{question}
Can this be applied to produce a rigid cover of $\pi_1(S^3 \setminus L)$ or of $S^3$ branched along $L$?
\end{question}

This is probably not so interesting.

\section{Realisation of Coxeter Groups}
\begin{question}
Let $W$ be a Coxeter group. Does there exist a link $L \subset S^3$ such that $W$ is a quotient of $\pi_1(S^3 \setminus L)$?
\end{question}

Yes. The trivial link.

\begin{proposition}
If $K \subset S^3$ is a knot and $W$ is a quotient of $\pi_1(S^3 \setminus K)$, then at most one edge in the Coxeter diagram of $W$ is labeled $2$.
\end{proposition}

\begin{proof}
If not then $\pi_1(S^3 \setminus K)$ has a quotient of the form $(\mathbb{Z}/2)^k$ for some $k \geq 2$. But the abelianization of the $\pi$-orbifold quotient is isomorphic to $\mathbb{Z}/2$, which does not have a quotient isomorphic to $(\mathbb{Z}/2\mathbb{Z})^k$.
\end{proof}

\end{document}
