\documentclass[main.tex]{subfiles}

\begin{document}

\section{Torus Knots}\label{sec:torus-knots}
It turns out that torus knots tend to admit few Coxeter quotients. This will be discussed in Section \ref{subsec:odd-weights}, where it is shown that an infinite class of torus knots does not admit any non-trivial Coxeter quotients, further adding to the potentially familiar statement that \textit{knots of determinant one admit poor representation theory}. As an upshot, in Section \ref{subsec:close-weights} we show that an infinite family of torus knots does admit high-rank Coxeter quotients.

\begin{proposition}
Let $p, q$ be positive integers. Then we have that the bridge index of the $(p, q)$-torus knot is $p$. In particular, the $(p, q)$-torus knot does not admit any Coxeter quotients of rank higher than~$p$.
\end{proposition}

This was first proved by Schubert himself \cite{schubert1954}.
There is also a more modern and potentially more accessible proof by Schultens~\cite{schultens2007}.

\subsection{Odd Weights}\label{subsec:odd-weights}
The qualitative features as far as Coxeter quotients go depend heavily on the weights $p, q$ of the $(p, q)$-torus knots. If we require $p$ and $q$ to be odd, we get particularly few Coxeter quotients. In fact, if $p$ and $q$ are odd, the $(p,q)$-torus knot can only have finite Coxeter quotients, as is hinted at in the proof of Theorem \ref{thm:p=3}. A further fact that leads us to expect few Coxeter quotients lies in the fact that if $p$ and $q$ are odd, then the $(p, q)$-torus knot has determinant one. To see this, consider the following.

\begin{theorem}
Let $p, q$ be odd positive integers. Then the Alexander polynomial of $K = T(p, q)$ is
$$\Delta_K(t) = \frac{(t^{pq}-1)(t-1)}{(t^q-1)(t^p-1)}.$$
\end{theorem}

A proof of this fact can be deduced from Example 9.15 in Burde and Zieschang's book \cite{burde2003}.
Because the determinant of a knot is its Alexander polynomial evaluated at $-1$, we immediately obtain that $\det T(p, q) = 1$ whenever $p$ and $q$ are odd.

\begin{corollary}
Let $p, q$ be odd. Then the $(p, q)$-torus has determinant one.
\end{corollary}

In Section \ref{subsec:rank-two} we have seen that this implies that torus knots with odd weights do not admit dihedral quotients. This brings us one step closer to the claim that torus knots with odd weights have very few Coxeter quotients. Let us first fix $p = 3$.

\begin{theorem}\label{thm:p=3}
Let $q$ be an odd positive integer such that $q$ has no prime factor less than or equal to~$5$. Then the $(3, q)$-torus knot does not admit any non-trivial Coxeter quotients.
\end{theorem}

Before proving this, we need an easy lemma.

\begin{lemma}\label{lem:p=3}
The only irreducible finite Coxeter groups of rank three with cyclic abelianization have graphs $A_3$ or $H_3$. In particular, any such Coxeter group has order $24$ or $120$.
\end{lemma}

\begin{proof}
This follows directly from the classification of finite irreducible Coxeter groups, see Table \ref{tab:classification}.
\end{proof}

\begin{proof}[Proof of Theorem \ref{thm:p=3}]
By Lemma \ref{lem:fox-coloring-and-determinant} it suffices to prove that $K = T(3, q)$ does not admit any rank three Coxeter quotients because $K$ has determinant one. Toward a contradiction, suppose $W$ is a Coxeter quotient of $K$ of rank three. Since $K$ is a knot we have that $W$ is irreducible. We will now consider two cases.

First, if $W$ is infinite, then $W$ has trivial center. Let $G = \pi(K)$ and consider the quotient map $\varphi: G \rightarrow W$. Recall that this means that for any meridian $m$ of $K$ we have that $\varphi(m)$ is a reflection in~$W$. Since $\varphi$ is surjective we have that the center of $G$ is mapped into the center of $W$, and is thus sent to the identity. Recall, e.g., from Rolfsen \cite{rolfsen2003}, that $G$ has a presentation
$$G = \langle a, b \; | \; a^3 = b^q \rangle$$
where $a$ and $b$ are meridian and longitude of the torus on which $K$ lies, respectively. Moreover, the center of $G$ is generated by the element $a^3 = b^q$, which is a composition of an odd number of diagram meridians of $K$ since both $3$ and~$q$ are odd. But this is a contradiction: the image of $a^3 = b^q$ in $W$ under~$\varphi$ is orientation-preserving because it is the identity, and orientation-reversing because it is the product of an odd number of reflections.

Finally, suppose that $W$ is finite. Then we have that $W$ is isomorphic to either $S_4$ or $\mathbb{Z}_2 \times A_5$ since~$K$ is a knot. Note that $\varphi(b)$ has order dividing $2q$, but by assumption $q$ has no divisor that is also a divisor of the order of $W$, which is either $24 = 2^3 \cdot 3$ or $120 = 2^3 \cdot 3 \cdot 5$. So $\varphi(b)$ has order $2$, but then $\varphi(b)$ is the unique orientation-reversing element of the center of $W$. Similarly, $\varphi(a)$ is the same element. But then $W$ is trivial.
%There are no infinite ones because $\pi(K)$ has non-trivial center and no finite ones because neither $S_4$ nor $\mathbb{Z}_2 \times A_5$ have elements of order the prime factor greater than or equal to $7$.
\end{proof}

A very similar Theorem is true for $p \geq 5$, as another close look at the classification of finite irreducible Coxeter groups in Table \ref{tab:classification} shows.

\begin{lemma}\label{lem:$p>=5$}
No irreducible finite Coxeter group of odd rank $p \geq 5$ has an element of order $q > p$, where~$q$ is prime.
\end{lemma}

\begin{theorem}
Let $5 \leq p < q$ be any odd coprime integers such that $q$ has no prime factor less than or equal to $p$. Then the $(p, q)$-torus knot does not admit any non-trivial Coxeter quotients.
\end{theorem}

\begin{proof}
This is completely analogous to the proof of Theorem \ref{thm:p=3}, using Lemma \ref{lem:$p>=5$} instead of \ref{lem:p=3}.
\end{proof}



\subsection{Close Weights}\label{subsec:close-weights}
This section contains an upshot showing that some torus knots in fact do have Coxeter quotients, namely the $(n, n + 1)$-torus knots.

\begin{example}[$n=4$]
Consider the braid representation of the $(4, 5)$-torus knot in Figure~\ref{fig:torus45}.

\begin{figure}[ht]
\centering
\includegraphics{t(4,5)}
\caption{The case $n = 4$}
\label{fig:torus45}
\end{figure}

Considering the first three strands we obtain a presentation of the reflection quotient of $K$, namely
$$r(K) = \langle a, b, c, d \; | \; adcbababcd, abadcbabcbabcd, abcbadcbabcdcbabcd \rangle^{(2)}.$$
The rules $a \mapsto (1 \; 2)$, $b \mapsto (2 \; 3)$, $c \mapsto (3 \; 4)$, $d \mapsto (4 \; 5)$ extend to a surjective homomorphism $r(K) \rightarrow S_5$, because all relators are mapped to the identity.
\end{example}

\begin{theorem}
Let $n \geq 2$ be an integer. Then the $(n, n+1)$-torus knot admits a Coxeter quotient isomorphic to $S_{n+1}$. In particular, $(n, n+1)$-torus knots are Coxeter.
\end{theorem}

\begin{proof} Similarly as before we consider the braid representation of the $(n, n+1)$-torus knot in Figure~\ref{fig:torusnn+1}. The first $n-1$ strands give us the following defining relations.
\begin{itemize}
\setlength\itemsep{0em}
\item $s_1 = s_n \cdots s_1 s_2 s_1 \cdots s_n$
\item $s_1s_2s_1 = s_n \cdots s_1 s_2 s_3 s_2 s_1 \cdots s_n$
\item \dots
\item $s_1 \cdots s_i \cdots s_1 = s_n \cdots s_1 \cdots s_{i+1} \cdots s_1 \cdots s_n$ for $1 \leq i \leq n-1$
\item \dots
\item $s_1 \cdots s_{n-1} \cdots s_1 = s_n \cdots s_1 \cdots s_n \cdots s_1 \cdots s_n$
\end{itemize}
These relations are all satisfied under the map $s_i \mapsto (i \;\;\; i+1)$. To see this, compute 
$$s_{i+1} \cdots s_1 \cdots s_{i+1} \cdots s_1 \cdots s_{i+1} \mapsto (1 \;\; i+1)$$
which is fixed by conjugation with $s_j$ for $j > i$. So the right hand side of the relation is equal to $(1 \; \; i + 1)$. But $s_1 \cdots s_{i-1} \mapsto (1 \; \cdots \; i)$, which conjugates $s_i$ to $(1 \; \; i + 1)$, so the left hand side is also equal to $(1 \; \; i + 1)$. 
\end{proof}

\begin{figure}
\centering
\includegraphics{t(n,n+1)}
\caption{The braid representation of the $(n, n+ 1)$-torus knot}
\label{fig:torusnn+1}
\end{figure}
%
%\section{Pretzel Knots}
%\subsection{Preliminaries}
%\subsection{A Coxeter Quotient}
%\begin{theorem}\label{thm:pretzel-links-are-coxeter}
%Pretzel links are Coxeter. 
%\end{theorem}
%
%\begin{proof} First consider the situation of how assigning letters to meridians propagates in 2-braids in Figure \ref{fig:elementary-twists}.
%\begin{figure}[ht]
%\centering
%\begin{minipage}{.35\textwidth}
%\centering
%
%\begin{tikzpicture}[scale = 0.6]
%\begin{knot}[clip width = 5, flip crossing = 2]
%
%\strand[black, thick] (0, 0) .. controls +(0, 1) and +(0, -1) .. (2, 2) .. controls +(0, 1) and +(0, -1) .. (0, 4);
%
%\strand[black, thick] (2, 0) .. controls +(0, 1) and +(0, -1) .. (0, 2) .. controls +(0, 1) and +(0, -1) .. (2, 4);
%
%\strand[black, thick] (0, 6) .. controls +(0, 1) and +(0, -1) .. (2, 8);
%
%\strand[black, thick] (2, 6) .. controls +(0, 1) and +(0, -1) .. (0, 8);
%
%\end{knot}
%
%\node at (1, 4.6) [circle, fill, inner sep=1pt]{};
%\node at (1, 5) [circle, fill, inner sep=1pt]{};
%\node at (1, 5.4) [circle, fill, inner sep=1pt]{};
%
%
%\node at (-1.4, 0) {$x$};
%\node at (3.4, 0) {$y$};
%\node at (-1.4, 2) {$xyx$};
%\node at (-1.4, 4) {$(xy)^2x$};
%\node at (-1.4, 6) {$(xy)^{n-1}x$};
%\node at (-1.4, 8) {$(xy)^nx$};
%\node at (3.4, 8) {$(xy)^{n-1}x$};
%
%\end{tikzpicture}
%
%\end{minipage}%
%\begin{minipage}{.35\textwidth}
%\centering
%\begin{tikzpicture}[scale = 0.6]
%\begin{knot}[clip width = 5, flip crossing/.list = {1, 3}]
%
%\strand[black, thick] (0, 0) .. controls +(0, 1) and +(0, -1) .. (2, 2) .. controls +(0, 1) and +(0, -1) .. (0, 4);
%
%\strand[black, thick] (2, 0) .. controls +(0, 1) and +(0, -1) .. (0, 2) .. controls +(0, 1) and +(0, -1) .. (2, 4);
%
%
%\strand[black, thick] (0, 6) .. controls +(0, 1) and +(0, -1) .. (2, 8);
%
%\strand[black, thick] (2, 6) .. controls +(0, 1) and +(0, -1) .. (0, 8);
%
%\end{knot}
%
%
%\node at (1, 4.6) [circle, fill, inner sep=1pt]{};
%\node at (1, 5) [circle, fill, inner sep=1pt]{};
%\node at (1, 5.4) [circle, fill, inner sep=1pt]{};
%
%\node at(-1.4, 0) {$z$};
%\node at(3.4, 0) {$w$};
%\node at (3.4, 2) {$wzw$};
%\node at (3.4, 4) {$(wz)^2w$};
%\node at (3.4, 6) {$(wz)^{n-1}w$};
%\node at (3.4, 8) {$(wz)^n w$};
%\node at (-1.4, 8) {$(wz)^{n-1}w$};
%
%\end{tikzpicture}
%
%\end{minipage}
%\caption{The relations for 2-braids}
%\label{fig:elementary-twists}
%\end{figure}
%
%Now note that assigning letters of meridians as in the local situation in Figure \ref{fig:local-picture-pretzel-link}, we in fact obtain a generating set of the link group $\pi(P(l_1, \dots, l_k))$.
%
%
%\begin{figure}[ht]
%\centering
%\begin{tikzpicture}
%\draw[ultra thick] (0, -1.5) -- (0, 0) -- (1, 0) -- (1, -1.5);
%\draw[ultra thick] (2, -1.5) -- (2, 0) -- (3, 0) -- (3, -1.5);
%
%\begin{knot}
%\strand[thick] (0.25, 0) .. controls +(0, 0.2) and +(0.2, 0) .. (-1, 0.2);
%\strand[thick] (0.75, 0) .. controls +(0, 0.2) and +(0, 0.2) .. (2.25, 0);
%\strand[thick] (2.75, 0) .. controls +(0, 0.2) and +(0.2, 0) .. (4, 0.2);
%\end{knot}
%
%\node at (0, 0.4) {$x$};
%\node at (1, 0.4) {$y$};
%\node at (2, 0.4) {$z$};
%\node at (3, 0.4) {$w$};
%
%\node at (0.5, -0.75) {$a$};
%\node at (2.5, -0.75) {$b$};
%\end{tikzpicture}
%
%\caption{A local picture}
%\label{fig:local-picture-pretzel-link}
%\end{figure}
%
%Putting these 2-braids together as in Figure \ref{fig:local-picture-pretzel-link} we obtain the following relation in the knot group from the requirement that $y = z$ according as to whether $a$ and $b$ are positive or negative:
%
%\begin{itemize}
%\setlength\itemsep{0em}
%\item If both $a$ and $b$ are positive, then we obtain $(xy)^{a-1}x = (yw)^by.$
%\item If $a$ is positive and $b$ is negative, we obtain $(xy)^{a-1}x = (wy)^{-b+1}w$.
%\item If $a$ is negative and $b$ is positive, we obtain $(yx)^{-a}y = (yw)^b y$.
%\item If both $a$ and $b$ are negative, then we obtain $(yx)^{-a}y = (wy)^{-b+1}w$.
%\end{itemize}
%
%All these relations follow from the Coxeter relations $(xy)^{|a|} = (yw)^{|b|} = 1$. Assigning generators as in Figure \ref{fig:local-picture-pretzel-link} we get a quotient of the desired form.
%\end{proof}
%
%\begin{corollary}
%Pretzel links satisfy the meridional rank conjecture.
%\end{corollary}
%%TODO cite
%
%\begin{proof}
%This is just Corollary \ref{cor:coxeter-links-meridional-rank}.
%\end{proof}
%
%It might be worth pointing out that the explicit Coxeter quotient obtained is described by the matrix
%
%$$
%\renewcommand\arraystretch{1.2}
%M = \left( \begin{matrix}
%1 & l_1 & \infty & \cdots & \cdots &\cdots & \infty & l_n \\
%l_1 & 1 & l_2 & \infty & \cdots & \cdots & \infty & \infty \\
%\infty & l_2 & 1 & l_3 & \infty &\cdots & \infty & \infty \\
%\vdots & \infty  & l_3 & 1 & \ddots & \ddots & \vdots & \vdots \\
%\vdots & \vdots & \infty & \ddots & \ddots & \ddots & \infty & \vdots \\
%\vdots & \vdots & \vdots & \ddots &\ddots & 1 & \l_{n-2} & \infty \\
%\infty & \infty & \infty & \cdots & \infty & l_{n-2} & 1 & l_{n-1} \\
%l_n & \infty & \infty & \cdots & \cdots & \infty  & l_{n-1} & 1
%\end{matrix} \right).$$
%
%\newpage

\end{document}