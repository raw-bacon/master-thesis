\documentclass[main.tex]{subfiles}

\begin{document}
\section{Coxeter Quotients} \label{sec:coxeter-quotients}
\subsection{The Reflection Quotient}\label{subsec:reflection-quotient}
Let $L \subset S^3$ be a link. A \textit{Coxeter quotient} of $L$ is a quotient $\pi(L)/N$ of the link group which is isomorphic to a Coxeter group $W$ such that every meridian in $\pi(L)$ corresponds to a reflection in $W$.
In many cases, finding Coxeter quotients of links is rather straight-forward, at least with the help of a computer. This is because checking whether a specific set of Coxeter relations induces a Coxeter quotient of a link amounts to checking the validity of the defining relations of its link group.

The \textit{reflection quotient} is the group $r(L) = \pi(L) / \langle \langle m^2 \rangle \rangle$, where $m$ is any meridian of $L$ and $\langle \langle m^2 \rangle \rangle$ denotes the normal subgroup of $\pi(L)$ generated by $m^2$. Because it is clumsy to write down a presentation in this fashion we agree to add the relations $s^2$ for all generators $s$ in $S$ when writing a presentation as $\langle S \; | \; R \rangle^{(2)}$. Using this notation, the reflection quotient can be defined to be
$$r(L) = \pi(L)^{(2)}.$$

Coxeter quotients of a link $L$ are also quotients of $r(L)$. An additional reason to focus on this group instead of $\pi(L)$ is that the amount of information lost by ignoring squares of meridians is small in most cases. This can be formally justified using the following.

\begin{theorem}[Boileau-Zimmermann \cite{boileau1989}, Theorem 1]
Let $K$ and $K'$ be prime knots such that $r(K)$ and $r(K')$ are infinite. Then $K$ and $K'$ are equivalent if and only if $r(K)$ and $r(K')$ are isomorphic.
\end{theorem}

%Be warned that in \cite{boileau1989} the reflection quotient is denoted $O(L)$ and is called the $\pi$-orbifold group.

It is easy to write down a presentation of the reflection quotient from a presentation of the link group $\pi(L) = \langle S \; | \; R \rangle$ of a link $L \subset S^3$. Just add the relations $s^2$ for all $s$ in $S$, and simplify $R$ by replacing all occurrences of $s^{-1}$ by $s$ and all occurrences of $s^2$ by the empty word. These presentations are generally shorter, easier to read and less prone to mistakes since it is no longer possible to confuse meridians with their inverses.
Note in addition that the Wirtinger algorithm (see Section \ref{subsec:link-group}) can be simplified so that the relation $ca = ab$ in Figure \ref{fig:wirtinger-relations} reads $acab = 1$ in the reflection quotient.
We will now mainly consider the reflection quotient instead of the link group.


\begin{figure}[ht]
\centering
\includegraphics{8_5}
\caption{The knot $8_5$}
\label{fig:8-5}
\end{figure}

\begin{example}[The Knot $8_5$]
Consider $K = 8_5$ from Rolfsen's knot table, compare Figure \ref{fig:8-5}. Then~$K$ has a Coxeter quotient with graph $A_3$, whose associated Coxeter group is isomorphic to the permutation group $S_4$.
Note that by the Wirtinger algorithm, the reflection quotient (defined in Section \ref{subsec:reflection-quotient}) has a presentation
$$r(8_5) = \langle a, b, c\; | \; babc(ba)^2bc(ba)^4, babc(ba)^4cbabc(ab)^2a \rangle^{(2)}.$$
These relations are satisfied in the Coxeter group
$$W = \langle a, b, c \; | \; (ab)^2, (ac)^3, (bc)^3 \rangle^{(2)}.$$
Thus, $W$ is a quotient of $\pi_1(S^3 \setminus 8_5)$.
\end{example}

Note that there exist presentations of the knot group of $8_5$  such that the above method does not work. For example, any presentation involving more than three meridians makes the method fail. This indicates that it is difficult to prove that a fixed link $L$ does not admit a Coxeter quotient isomorphic to a specific Coxeter group~$W$.

\subsection{Examples of Coxeter Knots}\label{subsec:list-of-quotients}
A \textit{Coxeter knot} is a knot $K$ admitting a Coxeter quotient of rank $b(K)$. In Theorem \ref{thm:c<=mu<=b} we prove that this implies that $K$ satisfies the meridional rank conjecture.
The following list of Coxeter knots in Rolfsen's table \cite{rolfsen2003} was found using the following algorithm. First, compute a presentation of the reflection quotient using the Wirtinger algorithm, modified such that the presentation does not use more generators than the bridge index (which is $3$ in all of the cases considered below) of the knot in question. Then, use some solution of the word problem in Coxeter groups to find all Coxeter quotients that map generating meridians to generating reflections. Note that this algorithm may not detect all Coxeter knots. A slight improvement can be obtained by considering multiple presentations, i.e., choosing different minimal generating sets of $\pi(L)$. This algorithm runs well for knots of few crossings, but takes infeasibly long for knots of crossing number $13$ or higher.

Because all $2$-bridge knots are Coxeter (see Section \ref{subsec:rank-two}), Table \ref{tab:cox-quotients} consists of all $3$-bridge knots in Rolfsen's table with up to nine crossings that were found to be Coxeter by the above algorithm.

\begin{table}[htb]
\centering
\begin{minipage}[t]{0.45\textwidth}
\centering
\begin{tabular}[t]{c|c}
Knot & Coxeter Quotients \\
\hline
$8_5$ & \includegraphics{dynkin-diagrams/a_3} \\
$8_{10}$ & \includegraphics{dynkin-diagrams/a_3} \\
$8_{15}$ & \includegraphics{dynkin-diagrams/a_3} \\
$8_{19}$ & \includegraphics{dynkin-diagrams/a_3} \\
$8_{20}$ & \includegraphics{dynkin-diagrams/a_3} \\
$8_{21}$ & \includegraphics{dynkin-diagrams/a_3} \\
$9_{16}$ & \includegraphics{dynkin-diagrams/a_3} \\
$9_{22}$ & \includegraphics{dynkin-diagrams/h_3} \\
$9_{24}$ & \includegraphics{dynkin-diagrams/a_3} \\
$9_{25}$ & \includegraphics{dynkin-diagrams/h_3} \\
$9_{28}$ & \includegraphics{dynkin-diagrams/a_3} \\
$9_{30}$ & \includegraphics{dynkin-diagrams/h_3} \\
\end{tabular}
\end{minipage}%
\begin{minipage}[t]{0.55\textwidth}
\centering
\begin{tabular}[t]{c|c}
Knot & Coxeter Quotients \\
\hline
\verticalcenter{$9_{35}$}\rule{0pt}{4ex} & \verticalcenter{\verticalcenter{\includegraphics{dynkin-diagrams/a_3}} and \verticalcenter{\includegraphics{dynkin-diagrams/triangle}}} \\
\verticalcenter{$9_{36}$} & \includegraphics{dynkin-diagrams/h_3} \\
\verticalcenter{$9_{37}$} &
\verticalcenter{\verticalcenter{\includegraphics{dynkin-diagrams/a_3}} and
\verticalcenter{\includegraphics{dynkin-diagrams/triangle}}} \\
\verticalcenter{$9_{40}$} & $\;\;\,$\verticalcenter{\includegraphics{dynkin-diagrams/a_3}} and \verticalcenter{\includegraphics{dynkin-diagrams/h_3}} \\
\verticalcenter{$9_{42}$} & \includegraphics{dynkin-diagrams/h_3} \\
\verticalcenter{$9_{43}$} & \includegraphics{dynkin-diagrams/h_3} \\
\verticalcenter{$9_{44}$} & \includegraphics{dynkin-diagrams/h_3} \\
\verticalcenter{$9_{45}$} & \includegraphics{dynkin-diagrams/h_3} \\
\verticalcenter{$9_{46}$} &
\verticalcenter{\verticalcenter{\includegraphics{dynkin-diagrams/a_3}} and
\verticalcenter{\includegraphics{dynkin-diagrams/triangle}}} \\
\verticalcenter{$9_{48}$} &
\verticalcenter{\verticalcenter{\includegraphics{dynkin-diagrams/a_3}} and
\verticalcenter{\includegraphics{dynkin-diagrams/triangle}}} \\
\end{tabular}
\end{minipage}
\caption{List of Coxeter knots with bridge index $3$}
\label{tab:cox-quotients}
\end{table}

The $3$-bridge knots that were not found to have a rank $3$ Coxeter quotient are $8_{16}, 8_{17}$, $8_{18}$, $9_{29}$, $9_{32}$, $9_{33}$, $9_{34}$, $9_{38}, 9_{39}, 9_{41}, 9_{47}$ and $9_{49}$. It remains open whether any of those knots admits a Coxeter quotient of rank $3$.
Data concerning knots of crossing number $10$ can be found in Appendix~\ref{sec:data}.

\subsection{Meridional Rank}\label{subsec:meridional-rank}
For a link $L$ we define its \textit{meridional rank} to be the least possible number of meridians of $L$ needed to generate the group $\pi(L)$. Here, a \textit{meridian} is any generator of $H_1(S^3 \setminus L) = \pi(L)^{\text{ab}}$.
We define the \textit{Coxeter rank} of a link $L$, abbreviated $c (L)$, to be the maximal number $n$ such that $L$ has a Coxeter quotient of rank $n$. Abbreviating the bridge index of $L$ by $b(L)$, and the meridional rank of $L$ by $\mu(L)$, we have the following.

\begin{theorem}\label{thm:c<=mu<=b}
Let $L \subset S^3$ be any link. Then we have $c(L) \leq \mu(L) \leq b(L)$.
\end{theorem}

\begin{proof}
The inequality $\mu( L ) \leq b(L)$ is Theorem \ref{thm:mr<=b}. Moreover, $c (L) \leq \mu( L )$, is a direct consequence of the fact that the number of generating reflections is determined by the Coxeter system. For a proof, see for example Lemma 2.1 in \cite{felikson2009}.
\end{proof}

The following conjecture was formulated by Cappell and Shaneson \cite{cappell1978}.

\begin{conjecture}[Meridional Rank Conjecture]
The meridional rank of a link is equal to its bridge index.
\end{conjecture}

\begin{remark}[Previous results] \label{rem:previous-results}
So far, the meridional rank conjecture was proven for the following classes of links.

\begin{itemize}
\itemsep0em
\item Generalized Montesinos links (Boileau-Zieschang \cite{zieschang1985}, 1985)
\item Torus links (Rost-Zieschang \cite{rost1987}, 1987)
\item Links of meridional rank two (Boileau-Zimmermann \cite{boileau1989}, 1989)
\item Generalized Montesinos links (Lustig-Moriah \cite{lustig1993}, 1991)
\item A large class of iterated torus knots (Cornwell-Hemminger \cite{cornwell2014}, 2014)
\item Links with meridional rank three whose two-fold branched
covers are graph manifolds (Boileau-Jang-Weidmann \cite{boileau2015}, 2015)
\item Knots whose exteriors are graph manifolds (Boileau-Dutra-Jang-Weidmann \cite{boileau2017}, 2017)
\item A family of knots with unknotting number one yet arbitrarily high
bridge number (Baader-Kjuchukova \cite{baader2017}, 2017)
\item Twisted links and arborescent
links associated to bipartite trees with even weights (Baader-Blair-Kjuchukova \cite{baader2019}, 2019)
\end{itemize}

The definitions of generalized Montesinos links given in \cite{zieschang1985} and \cite{lustig1993} do not agree.

\end{remark}

Let us refer to a link $L \subset S^3$ that has a Coxeter quotient of rank $n = b(L)$ as a \textit{Coxeter link}. Then we obtain the following from $c(L) \geq b(L)$ together with Theorem \ref{thm:c<=mu<=b}.


\begin{corollary}\label{cor:coxeter-links-meridional-rank}
Coxeter links satisfy the meridional rank conjecture.
\end{corollary}

Another application of the techniques we have developed so far lies in the following.

\begin{theorem}\label{thm:connected-sums-coxeter-rank}
Let $K$ and $K'$ be knots. Then $c (K\#K') = c (K) + c (K') - 1$.
\end{theorem}

\begin{proof}
Let $W$ and $W'$ be Coxeter quotients of $K$ and $K'$, respectively. First we choose diagrams of $K$ and $K'$ such that the generators of $\pi(K)$ and $\pi(K')$ that get mapped to generating reflections of $W$ and $W'$ are diagram meridians. Let $s \in \pi(K)$ and $s' \in \pi(K')$ each be one of those diagram meridians.

Now observe that $\pi(K\#K')$ is of the form $\pi(K) * \pi(K') / ss'$. Thus, we can explicitly write down a Coxeter quotient of $K\#K'$ as follows. Let $\Gamma$ and $\Gamma'$ be the Coxeter graphs associated to $W$ and $W'$. Then there is a Coxeter quotient of $K\#K'$ with graph
\begin{center}
\begin{tikzpicture}
\draw[dashed] (0.85, 0.15) -- (0.15, 0.85);
\draw[dashed] (-0.15, 0.85) -- (-0.85, 0.15);
\draw (0.8, 0) -- (-0.8, 0);

\node at (0, 1) {$s$};
\node at (-1, 0) {$\Gamma$};
\node at (1, 0) {$\Gamma'$};
\node at (0, 0.2) {$\infty$};
\end{tikzpicture}
\end{center}
which has rank $\text{rk } W + \text{rk } W' - 1$. This proves that $c(K\#K') \geq c(K) + c(K') -1$. The other inequality follows from the fact that any maximal Coxeter quotient of $K\#K'$ induces Coxeter quotients of $K$ and of $K'$ whose ranks add up to $c(K\#K') + 1$.
\end{proof}

\begin{corollary}
There are knots of arbitrarily high meridional rank.
\end{corollary}

\begin{proof}
Just take iterated connected sums of any two-bridge knot.
\end{proof}

\begin{corollary}
Connected sums of Coxeter knots are Coxeter knots.
\end{corollary}

\begin{proof}
Let $J, K \subset S^3$ be arbitrary knots. Then we have already shown that $b(J\#K) = b(J) + b(K) - 1$ and $c (J\#K) \leq b(J\#K)$. If $J$ and $K$ are both Coxeter knots then equality immediately follows from Theorem \ref{thm:connected-sums-coxeter-rank}.
\end{proof}

\subsection{Rank Two Quotients}\label{subsec:rank-two}
In this section we have a particularly close look at dihedral quotients of link groups. As we shall see, this is closely related to a popular notion in the literature, described for example in an article by Przytycki \cite{przytycki1998}. 

Let $L$ be a link. We will refer to a homomorphism $c: \pi(L) \rightarrow D_n$ mapping all meridians of $L$ to reflections in $D_n$ as a \textit{Fox $n$-coloring}. Note that a surjective Fox $n$-coloring is a dihedral quotient of $L$. We will call a Fox $n$-coloring $c$ \textit{non-trivial} if the image of $c$ contains more than one reflection in $D_n$, and we will call a link $L$ \textit{Fox $n$-colorable} if $L$ admits a non-trivial Fox $n$-coloring. Note that any link admits trivial Fox $n$-colorings for all $n$.

\begin{example}[Fox $3$-colorings]
Let $a, b, c$ be the reflections of $D_3$. Then a Fox $3$-coloring of $L$ is just a labeling of the arcs in a diagram of $L$ such that at any crossing either only one symbol appears or all three symbols appear.
The reader may convince themselves of the fact that the trefoil knot in Figure \ref{fig:trefoil} is Fox $3$-colorable, whereas the figure-eight knot in Figure \ref{fig:figure-8} is not.

\begin{figure}[ht]
\centering
\begin{minipage}{0.5\textwidth}
 \centering
\includegraphics{trefoil}
\caption{The Trefoil Knot}
\label{fig:trefoil}
\end{minipage}%
\begin{minipage}{0.5\textwidth}
\centering
\includegraphics{figure-8}
\caption{The Figure-Eight Knot}
\label{fig:figure-8}
\end{minipage}
\end{figure}
\end{example}

Let $n$ be a prime number. We will now present an algorithm to check whether a given link $L$ is $n$-colorable. First we make the following identification of the set of reflections with $\mathbb{Z}_n$. Write $$D_n = \langle a, b \; | \; (ab)^n \rangle^{(2)}.$$ Then the set $R$ of reflections is the set of elements of the form $(ab)^ka$. We will identify this reflection with the element $k$ in $\mathbb{Z}_p$.
Under this identification, the Wirtinger relation at any crossing reads $$2k = i + j,$$ see Figure \ref{fig:wirtinger-dihedral}. Indexing the set of diagram meridians and the set of crossings in some manner we can interpret this as a system of linear equations over $\mathbb{Z}$. Let us call the matrix of this system $\widehat{C}$. Note that this system admits trivial solutions, namely the ones assigning the same element of $\mathbb{Z}$ to all diagram meridians. Ignoring this set of solutions corresponds to deleting a row and a column of $\widehat{C}$ to obtain what we will call the \textit{coloring matrix} $C$.

\begin{figure}[ht]
\centering
\includegraphics{wirtinger-dihedral}
\caption{The Wirtinger relation at a crossing after identification of diagram meridians with $\mathbb{Z}_p$}
\label{fig:wirtinger-dihedral}
\end{figure}


\begin{lemma}\label{lem:fox-coloring-and-determinant}
Let $L$ be any link and $n$ be prime. Then $L$ is Fox $n$-colorable if and only if $n$ divides $\det C$, where $C$ is the coloring matrix of $L$.
\end{lemma}

\begin{proof}
The condition for the linear system described above to have non-trivial solutions is for $C$ to be singular over $\mathbb{Z}_n$, which is obviously equivalent to $n$ dividing $\det C$ as a matrix with integer coefficients.
\end{proof}

One can prove that the absolute value of the determinant of $C$ neither depends on the diagram, nor on the choice of which row and column to delete from $\widehat{C}$. For this reason, we will refer to the determinant $|\det C|$ simply as the \textit{determinant} of $L$, written $\det L$. In the literature the determinant of $L$ is usually defined to be its Alexander polynomial evaluated at $-1$, but it turns out that these definitions are actually equivalent \cite{livingston1993}.
It can be shown \cite{lickorish1997}
that the determinant of a 2-bridge link $L$ is never equal to one. This implies that~$L$ has a dihedral quotient. Thus, all 2-bridge links are Coxeter.

%\begin{example}
%Consider the Trefoil $K$ indexed as in Figure \ref{fig:trefoil-labeled-for-coloring}.
%\begin{figure}[ht]
%\centering
%\begin{tikzpicture}
%\clip (-2, -1.6) rectangle (2,2.7);
%\begin{knot}[clip width = 5, consider self intersections = true, flip crossing = 2]
%
%\strand[black, thick] (0,2) .. controls +(2.2,0) and +(120:-2.2) .. (210:2) .. controls +(120:2.2) and +(60:2.2) .. (-30:2) .. controls +(60:-2.2) and +(-2.2,0) .. (0,2);
%\end{knot}
%\node at (0, 2) {\contour{white}{$i$}};
%\node at (210:2) {\contour{white}{$k$}};
%\node at (-30:2) {\contour{white}{$j$}};
%
%\node at (0, -1.1) {$1$};
%\node at (-1, 0.6) {$3$};
%\node at (1, 0.6) {$2$};
%\end{tikzpicture}
%\caption{The Trefoil Knot with labels}
%\label{fig:trefoil-labeled-for-coloring}
%\end{figure}
%
%Then the linear equations read $2i = j + k$, $2j = i + k$ and $2k = i + j$, corresponding to the matrix
%$$\widehat{C} = \left(\begin{matrix}
%2 & -1 & -1 \\
%-1 & 2 & -1 \\
%-1 & -1 & 2
%\end{matrix}\right).$$
%Deleting from $\widehat{C}$ the last row and the last column we obtain the coloring matrix
%$$C = \left( \begin{matrix}
%2 & -1 \\
%-1 & 2
%\end{matrix} \right),$$
%yielding that $\det K = 3$. Note that this agrees with the previous observation that $K$ is Fox $3$-colorable.
%\end{example}





\subsection{Realization}
In this section we are going to show that many Coxeter groups are Coxeter quotients of knots. In fact, \textit{all} Coxeter groups are Coxeter quotients of links.

\begin{proposition}\label{prop:trivial-realization}
Let $L$ be the trivial link on $n$ components. Then the link group $\pi(L)$ is free on $n$ generators which are meridians. In particular, any Coxeter group $W$ is a Coxeter quotient of $\pi(L)$.
\end{proposition}

Since this is a little underwhelming we will try to do better. The following construction is essentially due to Brunner \cite{brunner1992}.

\begin{algo}[Brunner's Construction]\label{algo:brunner's-construction}
Let $\Gamma$ be any Coxeter graph, this time with edges labeled~$2$ for commuting generators and no edge for no relations (contrary to the convention above). If $\Gamma$ is not planar, choose a maximal planar subgraph instead of $\Gamma$ and fix an embedding of $\Gamma$ into the plane.

Now consider the dual graph $\Gamma^*$ of $\Gamma$ whose vertices are the faces of $\Gamma$, with edges weighted $m$ connecting vertices $s, t$ if and only if there is an edge labeled $m$ separating the faces of $\Gamma$ corresponding to $s, t$. Note that $\Gamma^*$ is not necessarily a simple graph, even if $\Gamma$ is (e.g., if $\Gamma$ is a tree).

We now interpret $\Gamma^*$ as a set of instructions how to construct a surface $S$ whose boundary will be a link $L(\Gamma^*)$ admitting a Coxeter quotient isomorphic to the Coxeter group corresponding to $\Gamma$. First, blow up the vertices of $\Gamma^*$ to disks. Now replace each edge labelled $m$ between disks by bands with $m$ twists. This procedure is illustrated in Figure \ref{fig:brunners-construction}.
\end{algo}

\begin{figure}[htb]
\centering
\begin{minipage}{0.37\textwidth}
\centering
\includegraphics{brunner-graph}
\end{minipage}%
\begin{minipage}{0.63\textwidth}
\centering
\includegraphics{brunner-link}
\end{minipage}
\caption{Brunner's construction producing from $\Gamma^*$ (left) a link (right)}
\label{fig:brunners-construction}
\end{figure}

Beware that knots constructed in this way will not be prime.

\begin{theorem}\label{thm:correctness-of-brunner}
The link constructed by Brunner's construction \textup{(Procedure \ref{algo:brunner's-construction})} starting with a graph $\Gamma$ yields a link that has a Coxeter quotient isomorphic to the Coxeter group $W$ corresponding to a graph $\Delta$, whose associated Coxeter group surjects onto the Coxeter group corresponding to $\Gamma$.
\end{theorem}

A proof of this can be found in Brunner's paper \cite{brunner1992}.
Theorem \ref{thm:correctness-of-brunner} is much more interesting than the construction in Proposition \ref{prop:trivial-realization}. Indeed, we are almost ready to answer the following question: \textit{What Coxeter groups are Coxeter quotients of knots?} But first, we need the following lemma.

\begin{lemma}\label{lem:tree->knot}
Let $\Gamma$ be a Coxeter graph with no cycles, i.e., a tree, whose edges are all odd. Then Brunner's construction \textup{(Procedure \ref{algo:brunner's-construction})} applied to $\Gamma$ yields a knot.
\end{lemma}

\begin{proof}
Note that the dual graph of a tree is a wedge of circles, that is, a graph with only one vertex. So Brunner's construction yields a disk with bands with an odd number of twists attached. We will prove by induction on the number of bands that the boundary of this surface $S$ has only one component.

First note that if there are no bands attached, the boundary of the disk is the unknot. This concludes the base case of the induction. Now  fix an innermost band $B$, and start at the point $p$ on the boundary of $S$ that lies immediately to the left of $B$. Let $q$ be the point on $S$ that lies immediately to the right of $B$. Then tracing the boundary as in Figure \ref{fig:tracing-band} shows that there exists a path from $p$ to $q$ that only passes through $B$.

\begin{figure}[htb]
\centering
\includegraphics{tracing-band}
\caption{Tracing a band arising from Brunner's construction on a tree. Note that the situation is the same if instead of one twist there are an odd number of twists in $B$.}
\label{fig:tracing-band}
\end{figure}

Contracting this path yields a surface with one band less. By the induction hypothesis the boundary of this is a knot. Thus, so is the boundary of $S$.
\end{proof}

\begin{theorem}
Let $W$ be a Coxeter group. Then $W$ is a Coxeter quotient of a knot if and only if the set $T$ of reflections in $W$ is a single conjugacy class.
\end{theorem}

\begin{proof}
Let $\Gamma$ be the graph of $W$, and let $\Gamma_0$ be the subgraph of $\Gamma$ only consisting of the odd-weighted edges as in Section \ref{subsec:conjugacy-classes}. Since all meridians in a knot group are conjugate it follows that if $\Gamma_0$ is disconnected, then $W$ is not a Coxeter quotient of a knot.

Conversely, if $\Gamma_0$ is connected, apply Brunner's construction to $\Gamma_0$. This is a knot by \ref{lem:tree->knot}. Moreover, it has a Coxeter quotient isomorphic to the Coxeter group $\widetilde{W}$ corresponding to the graph $\Gamma_0$ whose non-edges are replaced by edges labeled $\infty$. This group itself has a Coxeter quotient isomorphic to~$W$.
\end{proof}

\end{document}