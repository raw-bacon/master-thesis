\documentclass[main.tex]{subfiles}

\begin{document}
\newpage
\section{Coxeter Groups}\label{sec:coxeter-groups}
Coxeter groups are of interest for example in geometric group
theory, where they form a particularly nice family of examples. This viewpoint is explained in a book by Davis
\cite{davis2008}.
In this thesis we are going to exploit their nice properties 
and apply them to knot theoretic concepts such as the 
meridional rank, compare Section \ref{subsec:meridional-rank}.

In this section we are going to introduce the basics of 
the theory of Coxeter groups, limited to just what we will 
use afterwards. This is by no means a proper introduction; 
the reader completely unfamiliar with the theory of 
Coxeter groups might want to read Chapter 5 from 
Humphreys' book~\cite{humphreys1990} before reading 
what follows.

\subsection{Coxeter Matrices and Graphs}
Let $S$ be a finite set and let $M = (m_{st})$ be a symmetric 
$S \times S$ matrix with coefficients in the set of 
natural numbers, possibly $\infty$, satisfying $m_{ss} = 1$ 
for all $s \in S$, and $m_{st} \geq 2$ if $s \ne t$. 
Such a matrix is called a \textit{Coxeter matrix}. 
Consider the group presentation
$$W = 
\langle 
	S \; | \; (st)^{m_{st}} = 1 \text{ for } s,t\in S 
\rangle,$$
where $(st)^\infty = 1$ means that there is no relation involving $s$ and $t$.
Then the group $W$ is called the \textit{Coxeter group}
corresponding to $M$. The tuple $(W,S)$ is called a
\textit{Coxeter system} and the cardinality~$\#S$ is referred
to as its \textit{rank}. Since in this thesis Coxeter groups
always arise with a generating set, we blur the
distinction between Coxeter groups and Coxeter systems.

Another way to encode the coefficients $m_{st}$ is to use
them as weights in weighted undirected graphs, so-called
\textit{Coxeter graphs}. In this case, the nodes of the graph
associated to $W$ are the generators $S$, and the weight
between the nodes $s$ and $t$ is $m_{st}$. The following
shorthand notations are very common: If $m_{st} = 2$, then 
the edge between $s$ and $t$ is omitted. If $m_{st} = 3$, then 
the weight on the edge between~$s$ and $t$ is omitted. 
Finally, if $m_{st} = 4$ then instead of a labeled line we 
draw an unlabeled double line.

\begin{example}[Dihedral Groups]
	The symmetry group of a regular $n$-gon is called the 
	\textit{dihedral group} of order $2n$, denoted by the 
	symbol $D_n$. This group is generated by two reflections
	$s, t$ whose composition is a rotation by an angle
	of~$2\pi/n$. This is used to show that $D_n$ 
	is isomorphic to the Coxeter group
		$$W = 
			\langle 
				s, t \; | \; (st)^n = s^2 = t^2 = 1 
			\rangle.$$
		Thus, Coxeter groups on two generators whose product
		has finite order are dihedral groups.

	This observation is often used to motivate the 
	introduction of an additional dihedral group, the so-called
	\textit{infinite dihedral group} with presentation
		$$D_\infty = 
			\langle 
				s, t \; | \; s^2 = t^2 = 1 
			\rangle.$$
\end{example}

\subsection{The Reflection Representation}
A well-known construction leading to an easy solution of the
word-problem for Coxeter groups and also yielding a
geometric viewpoint for the study of Coxeter groups is a faithful
representation of any Coxeter group where its generating set
acts as a set of reflections with respect to a (possibly
degenerate) bilinear form.

Let $(W,S)$ be a Coxeter system. Then we consider the bilinear
form on the vector space $V = \mathbb{R}^S$ given by the matrix
$B = (b_{st})$ with entries $b_{st} = -\cos \pi/m_{st}$. 
For a generating reflection $s$ in $S$ let $\rho_s$ be the
\textit{reflection} with respect to $B$ with root the basis
vector~$e_s$ in $\mathbb{R}^S$ given by the formula
$$ \rho_s(x) = x - 2B(x, e_s)e_s,$$
where $x$ is a vector in $V$.

\begin{proposition}
\label{prop:reflection-representation-well-defined}
Let $(W,S)$ be a Coxeter system and let $s, t \in S$. 
Then the reflections $\rho_s$ and $\rho_t$ satisfy 
$(\rho_s \rho_t)^{m_{st}} = 1$.
\end{proposition}

\begin{proof}
Let $n$ be the number of elements of $S$. Observe that each
reflection $\rho_s$ fixes an $(n-1)$-dimensional hyperplane,
namely the orthogonal complement of the non-isotropic vector
$e_s$. Now fix generating reflections $s$ and $t$ in $S$ and 
consider the subspace 
$V_{st} = \mathbb{R}e_s \oplus \mathbb{R}e_t$,
which is invariant under the maps $\rho_s$ and $\rho_t$. 
Since the orthogonal complement of $V_{st}$ is fixed by $\rho_s$ 
and $\rho_t$, the order of $\rho_s\rho_t$ on $V$ is equal to its 
order on $V_{st}$.

Let us now distinguish two cases. If $m_{st}$ is finite, $B$ 
restricted to $V_{st}$ is positive definite. To see this, 
consider an arbitrary non-zero vector $v = ae_s + b e_t$ in 
$V_{st}$ and compute
\begin{align*}
B(v, v) & = a^2B(e_s, e_s) + 2abB(e_s, e_t) + b^2B(e_t, e_t) \\
		& = a^2 + 2ab\text{cos}( -\pi / m_{st} ) + b^2 \\
		& > 0.
\end{align*}
Thus, $V_{st}$ is isometric to $\mathbb{R}^2$ with the standard 
inner product, and the angle between the invariant subspaces of 
$\rho_s$ and $\rho_t$ is equal to $2 \pi / m_{st}$. This shows 
that $\rho_s$ and $\rho_t$ are symmetries of a regular polygon 
and their product is a rotation by $2\pi/m_{st}$, which has 
order $m_{st}$.
On the other hand, if $m_{st} = \infty$ the matrix of 
$\rho_s\rho_t$ is given by
$$
\rho_s \rho_t = 
\left( \begin{matrix}
	-1 & 1 \\ 1 & -1
\end{matrix} \right)
\left( \begin{matrix}
	1 & -1 \\ -1 & 1
\end{matrix} \right)
= 
\left( \begin{matrix}
	-2 & 2 \\ 2 & -2
\end{matrix} \right)
$$
which has infinite order.
\end{proof}

This shows that the map $s \mapsto \rho_s$ induces a 
representation of $W$, called the 
\textit{reflection representation} 
of $W$. We will conclude this section by showing that this 
representation is faithful. But first we need some preparation.
To any Coxeter group $W$ we can assign a 
\textit{length function} 
$l: W \rightarrow \mathbb{N}$, 
assigning to any element~$w$ of $W$ the minimal number of 
generating reflections needed to write down an expression 
for $w$. An expression for $w$ of length $l(w)$ is called a 
\textit{reduced expression} for $w$.
To get a little bit of a feeling how the length function behaves 
for Coxeter groups (or more generally, for groups generated by 
elements of order two subject only to relations of even length, 
such as the reflection quotient from Section 
\ref{subsec:reflection-quotient}), we prove the following 
illustrative result.

\begin{proposition}
	Let $W$ be a Coxeter group, $w \in W$ and $s \in S$. 
	Then $l(ws) = l(w) \pm 1$.
\end{proposition}

\begin{proof}
First of all, we make the observation that for any words 
$w, w' \in W$ we have $l(ww') \leq l(w) + l(w')$. This 
immediately implies that $l(ww') \geq l(w) - l(w')$ by applying 
the above to the words $ww'$ and $w'^{-1}$ instead of $w$ and 
$w'$, observing that $l(w) = l(w^{-1})$ for all $w$ in $W$. We 
conclude that
$$l(w) - 1 \leq l(ws) \leq l(w) + 1.$$
The only thing left to show is that in fact $l(ws) \neq l(w)$. 
To see this, we introduce the \textit{sign} homomorphism $
\varepsilon: W \rightarrow \{\pm 1\}$ defined via $s \mapsto -1$ 
for all $s$ in $S$. Indeed, this extends to~$W$ since all 
relations of~$W$ are of even length. Obviously we have 
$\varepsilon(w) = (-1)^{l(w)}$, and 
$\varepsilon(ws) \neq \varepsilon(w)$. 
Using this, we conclude that $l(ws) \neq l(w)$, which is 
precisely what we wanted to show.
\end{proof}

For a non-zero vector $v$ in $V$ we will say that $v$ is 
\textit{positive} if $v$ is a non-negative linear combination of 
the basis vectors $e_s$. Moreover, $v$ is \textit{negative} if 
$-v$ is positive. This enables us to formulate the following 
important Lemma relating the length function to the reflection 
representation.

\begin{lemma}\label{lem:humphreys5.4}
	Let $W$ be a Coxeter group, $w$ a word in $W$ and $s$ a 
	generating reflection in $S$. Then the vector $we_s$ in 
	$V$ is positive if and only if we have that 
	$l(ws) = l(w) + 1$.
\end{lemma}

\begin{proof}
This is taken from \cite{humphreys1990}. Note first that it 
suffices to prove that whenever $l(ws) = l(w) + 1$ we have that 
$we_s$ is positive, since applying this to $ws$ instead of $w$ 
proves its converse. 
The proof of this implication is a proof by induction on $l(w)$. 
Suppose $l(w) = 0$. Then $w$ is the identity and we have that 
$l(ws) = l(w) + 1$ for all $s$. 

For the inductive step, let $t$ be any reflection in $S$ such 
that $l(wt) = l(w) - 1$, taking for example~$t$ to be the last 
symbol in a reduced expression for $w$. Write $w = w'w_{st}$, 
where $w_{st}$ is a word in the symbols $s, t$ of maximal length 
such that the length of the expression $w'w_{st}$ is equal to 
$l(w)$. This exists since $w_{st}$ equal to the empty word is in 
fact a (non-optimal) solution.
We will now prove the Lemma by showing that the vectors $w'e_s$, 
$w'e_t$ and $w_{st}e_s$ are all positive, and that $w_{st}$ is 
non-trivial. Taking $w_{st} = t$ shows that $l(w') < l(w)$, so 
we can apply the induction hypothesis to $w'$, showing that the 
vectors $w'e_s$ and $w'e_t$ are both positive, provided we can 
show $l(w's) > l(w')$ and $l(w't) > l(w')$. 
Suppose toward a contradiction that $l(w's) < l(w')$. Then 
$sw_{st}$ is in fact a longer word than $w_{st}$, contradicting the
maximality of $w_{st}$. Similarly it is also true that 
$l(w't) > l(w')$.

\begin{figure}[ht]
	\centering
	\begin{minipage}{.5\textwidth}
		\centering		
		\includegraphics{figs/pentagon}
	\end{minipage}%
	\begin{minipage}{.5\textwidth}
		\centering
		\includegraphics{figs/hexagon}
	\end{minipage}
	\caption{Illustration of the fact that $e_s$ lands in 
	the positive cone, indicated in grey. Note that composing
	reflections in the mirrors $e_s^\perp$ and $e_t^\perp$
	results in a rotation by an angle of $2\pi/m_{st}$.}
	\label{fig:humphreys-5.4-proof}
\end{figure}

Finally we draw a picture to show that $v_{st}e_s$ is positive.
Note first that $v_{st}$ does not have a reduced expression 
ending in $s$, since if it did we would actually reduce its 
length by adding an $s$ to the end. In particular, the length of 
$w_{st}$ is strictly less than $m_{st}$, that is, interpreting 
$w_{st}$ as a map on the plane as in the proof of Proposition 
\ref{prop:reflection-representation-well-defined}, $w_{st}$ is 
not rotation by an angle of $\pi$. The claim now follows from 
the fact that $w_{st}$ is a rotation by at most $\pi - \pi/m_{st}$, 
perhaps followed by a reflection in $e_t^\perp$, but only if 
this does not result in a rotation by an angle of~$\pi$. Also 
compare Figure \ref{fig:humphreys-5.4-proof}.
\end{proof}

Having established how the length function is related to the 
action of $W$ on the basis vectors of its geometric 
representation, we immediately obtain what we desired.

\begin{theorem}
	The reflection representation is faithful.
\end{theorem}

\begin{proof}
	Suppose $w$ lies in the kernel of the reflection
	representation. 
	Then obviously $we_s > 0$ for all generating 
	reflections $s$ in $S$, so by 
	Lemma \ref{lem:humphreys5.4}, 
	$l(ws) = l(w) + 1$ for 
	all $s$. That is, for any $s$ we have that $w$ does not 
	have a 
	reduced expression ending in $s$. But this implies that 
	$w$ is the identity.
\end{proof}

\subsection{Roots}\label{subsec:roots}
The main goal of this section is to show that Coxeter groups 
have tiny center, that is, the center of an infinite Coxeter 
group is trivial. To do this, our main tool will be the action 
of a Coxeter group on a specific subset of $V$ called roots.

The set of \textit{roots} of a Coxeter group $W$, denoted by the 
letter $\Phi$, is the set of all images of the basis vectors 
$e_s$ under $W$. The geometric observation justifying this 
terminology is the following. Let $w$ be an arbitrary element of 
$W$. Then, for all generating reflections $s$ in $S$, the map 
$wsw^{-1}$ acts on $V$ by a reflection whose eigenvector of 
eigenvalue one is the vector $we_s$.

Note that by Lemma \ref{lem:humphreys5.4}, every root is either 
positive or negative. Writing $\Pi$ for the set of positive 
roots, we can express this in symbols as $\Phi = \Pi \cup -\Pi$. 
This will be useful for a geometric characterization of the 
length function $l$ on a Coxeter group. Consider the function 
$n$ mapping an element $w \in W$ to the number of positive roots 
sent to negative roots by the action of $w$ on $V$.

\begin{proposition}\label{prop:n=l}
	Let $W$ be a Coxeter group. Then, for any element $w$ of 
	$W$ we have that $n(w) = l(w)$.
\end{proposition}

\begin{proof}
Let $s$ be a generating reflection in $S$. We first prove that the only positive root mapped to a negative root by $s$ is the basis vector $e_s$. Choose any positive root $\alpha \neq e_s$. Since $\alpha$ is not equal to $e_s$ there exists a reflection $t$ in $S$ such that the coefficient of $\alpha$ corresponding to the $e_t$-component of $\alpha$ is non-zero. But the reflection $s$ does not change the $e_t$-component of $\alpha$, so $s\alpha$ must be positive.

We now proceed by induction on the number $l(w)$. Suppose $l(ws) > l(w)$, and that $n(w) = l(w)$. By Lemma \ref{lem:humphreys5.4} it follows that $we_s$ is positive, which shows that $wse_s = -we_s$ is negative. Note that since $w$ induces a bijection on the remaining positive roots $\Pi \setminus \{e_s\}$, we must have $n(ws) = n(w) + 1 = l(ws)$.
\end{proof}

Recall that the \textit{radical} $V^\perp$ of the bilinear form $B$ on $V$ is the set of vectors~$v'$ in $V$ such that we have $B(v, v') = 0$ for all $v$ in $V$. Note that $V^\perp$ is not affected by any reflection, so in particular $V^\perp$ is a $W$-invariant subspace. We will now prove a partial converse to this for \textit{irreducible} Coxeter groups, that is, Coxeter groups whose associated Coxeter graph is connected.

\begin{theorem}\label{thm:invariant-subspaces}
	Let $W$ be an irreducible Coxeter group. Then, every 
	proper $W$-invariant subspace of~$V$ is contained 
	in $V^\perp$.
\end{theorem}

\begin{proof}
Let $V'$ be a $W$-invariant subspace of $V$. Suppose first that some basis vector $e_s$ is contained in~$V'$. Let $t$ be adjacent to $s$ in the Coxeter graph of $W$. Then $te_s$ has a non-zero $e_t$-component and is contained in $V'$, so it follows that also $e_t$ is in $V'$. Proceeding like this through the connected Coxeter graph of $W$ we obtain that $V'$ contains a basis of $V$. Therefore, $V'$ is not a proper subspace.

We have established that all proper $W$-invariant subspaces of $V$ do not contain any basis vectors~$e_s$. Let $V'$ be such a subspace. Then $V'$ is contained in an intersection of eigenspaces of the generating reflections $s$ in $S$. But since the proper subspace $V'$ does not contain the basis vectors~$e_s$, we have that~$V'$ must be contained in the eigenspaces of $s$ in $S$ of eigenvalue $1$, in other words, in the orthogonal complements of the basis vectors $e_s$.
\end{proof}

\begin{corollary}
	The center of an infinite Coxeter group is trivial.
\end{corollary}

\begin{proof}
Let $A$ be any endomorphism of $V$ commuting with every element of $W$. We will show that $A$ is actually multiplication with a scalar. Let $s \in S$ be any generating reflection. Since $A$ commutes with $s$ their eigenspaces must agree. In particular, the line $\mathbb{R}e_s$ is an eigenspace of $A$ of eigenvalue $c$ for some~$c$. 

Let $V'$ be the kernel of the map $A-c$. Let $v \in V'$. Then $Av = cv$. But then we have
\begin{align*}
	(A-c)wv & = Awv - cwv \\
			& = wAv - cwv \\
			& = w(A - c) v \\
			& = 0,
\end{align*}
so $V'$ is $W$-invariant. Since $V'$ contains $\mathbb{R}e_s$ 
and $\mathbb{R}e_s$ is not contained in $V^\perp$, we 
obtain by Theorem~\ref{thm:invariant-subspaces} that $V' = V$, 
in other words, that $A$ is multiplication with $c$.

We now consider what happens if $A$ is actually an element of 
$W$. If $c > 0$ then all roots in $\Pi$ are mapped to positive 
roots, so by Lemma \ref{lem:humphreys5.4} we have that $A$ is 
the identity. If $c < 0$ we first make the following 
observation: $\Pi$ is finite if and only if $W$ is. Now observe 
that all positive roots in $\Pi$ are mapped to negative roots, 
so the length of $A$ as an element of $W$ is infinite by Lemma 
\ref{lem:humphreys5.4}, which is absurd.
\end{proof}

\subsection{Conjugacy Classes}\label{subsec:conjugacy-classes}
Imposing as the title of this section may sound, we are not 
going to be able to give a description of how conjugacy classes 
work in Coxeter groups. We will however be able to determine the 
number of conjugacy classes in the set of reflections.

Let $W$ be a Coxeter group generated by the set $S$. A 
\textit{reflection} is an element $w \in W$ that is conjugate to 
an element in $S$. We will write $T$ for the set of reflections.

\begin{proposition}\label{prop:conjugacy}
	Let $\Gamma$ be the Coxeter graph of a Coxeter group $W$,
	and let $s, t \in S$. Then $s$ and $t$ are conjugate if and
	only if there exists a path in $\Gamma$ from $s$ to $t$
	whose edges only consist of odd weights.
\end{proposition}

\begin{proof}
Recall that being conjugate is an equivalence relation. By 
transitivity it suffices to prove, for sufficiency, that if 
there exists a relation $(st)^{m_{st}}$ for odd $m_{st}$, then $s$ 
and $t$ are conjugate. But with this relation we have 
for $k = (m_{st}-1)/2$ that $t = (st)^ks(ts)^k$.

We prove the converse by contraposition. If there exists no 
such path between $s$ and $t$, the subgraph~$\Gamma_0$ of 
$\Gamma$ consisting of the same vertices but only of the odd
weighted edges is disconnected. Moreover, $s$ and $t$ lie in
distinct path-components of $\Gamma_0$. Let 
$\varepsilon: W \rightarrow \{\pm1\}$ be the homomorphism
induced by sending each reflection in $S$ that lies in the 
same component as $s$ to $-1$, and each of the other reflections 
to $1$. This is well-defined. To see this, note that the weights 
between two elements in~$S$ that are not sent to the same 
element under $\varepsilon$ are always even, so the defining relators 
in $W$ indeed lie in the kernel of $\varepsilon$ considered as a 
map on the free group $F(S)$.

Thus, $\varepsilon$ is a homomorphism to an abelian group 
mapping $s$ and $t$ to distinct elements. But since the image of 
a conjugacy class lies in one conjugacy class, this shows that 
$s$ and $t$ are not conjugate.
\end{proof}

Let $\Gamma_0$ be as in the proof of Proposition 
\ref{prop:conjugacy}. Then we can restate the above as follows: 
\textit{There is exactly one conjugacy class in $T$ for each 
component of $\Gamma_0$}. In particular, all reflections in $W$ 
are conjugate if and only if $\Gamma_0$ is connected.

\subsection{The Word Problem}
The reflection representation gives us a way to write down elements of a Coxeter group numerically, in effect solving the word problem, whicih asks whether a given element of a group is equal to the identity. This section outlines another, more syntactic approach to the word problem. The described procedure was found by Tits \cite{tits1969}.

\begin{notation}
Let $W$ be a Coxeter group with generating set $S$ and Coxeter matrix $M = (m_{st})$. Then we agree on the convention that
$$(st)^{m_{st}/2} =  \begin{cases}
(st)^k & \text{ if } m_{st} = 2k, \\
(st)^ls & \text{ if } m_{st} = 2l + 1.
\end{cases}$$
E.g., if $m_{st}=5$ we write $(st)^{5/2} = ststs$. 
\end{notation}

\begin{theorem}[Tits]\label{thm:word-problem}
Let $F$ be the free monoid generated by the set $S$. Suppose the word $w~\in~F$ represents the trivial element in a Coxeter group $W$ with generating set $S$ and Coxeter matrix $M~=~(m_{st})$. Then there is a sequence of moves of the following two types carrying $w$ to the empty word.

\begin{enumerate}[\textup{(}i\textup{)}]
\setlength\itemsep{0em}
\item removing occurrences of $s^2$ from $w$ for $s \in S$.
\item replacing occurrences of $(st)^{m_{st}/2}$ by $(ts)^{m_{st}/2}$ for $s, t \in S$.
\end{enumerate}
In particular, there is a sequence of the above moves carrying $w$ to the empty word without making $w$ longer.
\end{theorem}

\subsection{Finite Coxeter Groups}
Finite Coxeter groups differ from infinite ones in a few key ways. For example, with the techniques introduced in Section \ref{subsec:roots} we can prove the following interesting result.

\begin{theorem}
Let $W$ be a finite Coxeter group. Then $W$ has a unique longest element $w_0$.
\end{theorem}

\begin{proof}
Note that by Proposition \ref{prop:n=l} we have that the length function is bounded from above by the cardinality $|\Pi|$. Moreover, if $w \in W$ sends all $e_s$ to negative roots, then $we_s$ is a negative linear combination of basis vectors, whence $w$ sends all positive roots to negative roots. In other words, if a word $w \in W$ is not of length $|\Pi|$, then there exists a basis vector $e_s$ not sent to a negative root. But then, by Lemma \ref{lem:humphreys5.4} the word $ws$ is longer than $w$. This shows that the length $|\Pi|$ is actually attained by some word $w_0 \in W$.

If $w_1$ is another word of maximal length, we have that $w_0w_1$ maps all positive roots to positive roots, so by Proposition \ref{prop:n=l} we have that its length is zero, i.e., $w_0w_1$ is the identity. It follows that the words $w_0$ and $w_1$ are equal.
\end{proof}

It turns out that finite Coxeter groups can be completely classified. This feat was first accomplished by Coxeter \cite{coxeter1935} himself. The classification of finite irreducible Coxeter groups will be useful for us in Section \ref{sec:torus-knots}. Since this is a well-documented exercise in combinatorial graph theory, see \cite{humphreys1990}, we will omit the proof and only state the theorem.

\begin{theorem}
Any finite irreducible Coxeter group can be found in Table \ref{tab:classification}.
\end{theorem}

\begin{table}[ht]
\centering
\begin{tabular}{c|c|c}
Coxeter Graph & Name & Size \\
\hline
\begin{tabular}{l}
\includegraphics{figs/dynkin-diagrams/a_n}
\end{tabular} & \begin{tabular}{l} $A_n$ \end{tabular} & \begin{tabular}{l}$(n+1)!$\end{tabular} \\
\begin{tabular}{l}
\includegraphics{figs/dynkin-diagrams/b_n}
\end{tabular}& \begin{tabular}{l}$B_n$\end{tabular} & \begin{tabular}{l}$2^n n!$\end{tabular} \\
\begin{tabular}{l}
\includegraphics{figs/dynkin-diagrams/d_n}
\end{tabular}& \begin{tabular}{l}$D_n$\end{tabular} & \begin{tabular}{l}$2^{n-1} n!$\end{tabular} \\
\begin{tabular}{l}
\includegraphics{figs/dynkin-diagrams/e_6}
\end{tabular}& \begin{tabular}{l}$E_6$\end{tabular} & \begin{tabular}{l}$72 \cdot 6!$\end{tabular} \\
\begin{tabular}{l}
\includegraphics{figs/dynkin-diagrams/e_7}
\end{tabular}& \begin{tabular}{l}$E_7$\end{tabular} & \begin{tabular}{l}$72 \cdot 8!$\end{tabular} \\
\begin{tabular}{l}
\includegraphics{figs/dynkin-diagrams/e_8}
\end{tabular}& \begin{tabular}{l}$E_8$\end{tabular} & \begin{tabular}{l}$192 \cdot 10!$\end{tabular} \\
\begin{tabular}{l}
\includegraphics{figs/dynkin-diagrams/f_4}
\end{tabular}& \begin{tabular}{l}$F_4$\end{tabular} & \begin{tabular}{l}$1152$\end{tabular} \\
\begin{tabular}{l}
\includegraphics{figs/dynkin-diagrams/h_3}
\end{tabular}& \begin{tabular}{l}$H_3$\end{tabular} & \begin{tabular}{l}$120$\end{tabular} \\
\begin{tabular}{l}
\includegraphics{figs/dynkin-diagrams/h_4}
\end{tabular}& \begin{tabular}{l}$H_4$\end{tabular} & \begin{tabular}{l}$14400$\end{tabular} \\
\begin{tabular}{l}
\includegraphics{figs/dynkin-diagrams/i_2(n)}
\end{tabular}& \begin{tabular}{l}$I_2(n)$\end{tabular} & \begin{tabular}{l}$2n$\end{tabular}
\end{tabular}
\caption{Classification of Irreducible Finite Coxeter Groups}
\label{tab:classification}
\end{table}
\newpage

\end{document}